%% 
%% Dissertation
%% 

%% � - find-replace these quotes as they break things


\documentclass[a4paper,12pt,twoside,openright]{report}


%%
%% Definitions of names and suchlike
%%

\def\authorname{David J.\ Barker\xspace}
\def\authorcollege{Jesus College\xspace}
\def\authoremail{David.Barker@cl.cam.ac.uk}
\def\dissertationtitle{Developing a formally verified algorithm for static register allocation}
\def\wordcount{TBC}


\usepackage{epsfig,graphicx,parskip,setspace,tabularx,xspace,holtexbasic,alltt,proof}

%% START OF DOCUMENT
\begin{document}


%% FRONTMATTER (TITLE PAGE, DECLARATION, ABSTRACT, ETC)
\pagestyle{empty}
\singlespacing
\input{titlepage}
\onehalfspacing
\input{declaration}
\singlespacing
\newpage
{\Huge \bf Abstract}
\vspace{24pt} 


This dissertation describes the implementation and verification of a full register allocation algorithm, designed to be representative of the sort of allocator that might be used in a practical optimising compiler. A model of three-address code (and its execution behaviour) was developed and a verified live variable analysis function and clash graph generator were constructed. A number of different graph colouring algorithms were then implemented and proved correct, as well as various heuristics. The work was then tied together by an overarching proof of correctness which demonstrates that the algorithm does not change code behaviour. The algorithm was also extended to construct preference graphs based on `move' instructions and use them in colouring, and to handle spilling of registers where not enough physical registers are available, and these additions were also verified in depth.


\newpage
\vspace*{\fill}


\pagenumbering{roman}
\setcounter{page}{0}
\pagestyle{plain}
\tableofcontents
\listoffigures
\listoftables

\onehalfspacing

%% START OF MAIN TEXT 

\chapter{Introduction}
\pagenumbering{arabic}
\setcounter{page}{1}

This is the introduction where you should introduce your work.  In
general the thing to aim for here is to describe a little bit of the
context for your work --- why did you do it (motivation), what was the
hoped-for outcome (aims) --- as well as trying to give a brief
overview of what you actually did.

It's often useful to bring forward some ``highlights'' into 
this chapter (e.g.\ some particularly compelling results, or 
a particularly interesting finding). 

It's also traditional to give an outline of the rest of the
document, although without care this can appear formulaic 
and tedious. Your call. 


\chapter{Background} 

\section {The problem of register allocation}

When compiling code, we generally assume the existence of an infinite number of
locations in which to store values. Each variable used by the program is treated
as its own location, and temporaries and results are freely allocated to conceptual
registers. However, in reality all processors have only finite numbers of physical
registers, some of which are reserved for specific purposes, and so the compiler
ultimately has to map variables in the code to registers in such a way that ensures
the code's behaviour is the same.

There are many issues involved when doing this. The main problem is that code will
often use far more `virtual' registers than actually exist on the target processor.
The simplest solution to this is to store any values which won't fit in memory, but
this considerably worsens performance. The solution is to find variables which are
never in use at the same time and place them in the same register, allowing registers
to be re-used when the values originally assigned to them are not in use.

This optimisation is enabled by liveness analysis. Essentially, a variable is live
where changing its value will affect the program's future behaviour [ref?]. Because
this is difficult to decide with absolute precision, we generally compute a safe
over-approximation of liveness: a variable is live if an instruction reachable from
here reads its value, and its value is not written to in a preceding instruction.
Liveness is computed by repeatedly iterating over instructions in the code from the
end backwards, updating each instruction's set of live variables according the
following liveness equation:

\begin{center}
$live(n) = \left(live(n+1) \setminus write(n)\right) \cup read(n)$
\end{center}

(Where live(n) is the set of variables live at instruction n, live(n+1) is those
live at the next instruction and read(n) and write(n) are the variables read and written
by the instruction.)

This is repeated until the live variable sets stop changing. With this information,
we can deduce which variables are live at the same time, and thus which variables
cannot occupy the same register because doing so would result in one overwriting the
other whilst the other was about to be used.

When no allocation can satisfy these clashes between variables, it becomes necessary
to spill variables to memory. This means reading or writing said variables takes
considerably longer, and so an effective register allocation algorithm should minimise
spills wherever possible.

%TODO: [Find a reference or two for this next bit?]
Another thing to consider is that some target architectures have constraints on what
registers can be used for. For instance, a particular operator may require its operands
to be stored in particular registers, or may place the output in a designated `result'
register. This means move instructions will be required to satisfy these requirements,
slowing the overall program down. Good register allocators allow for this by biasing
allocation so that these move instructions can be eliminated wherever possible.

\section{Isomorphism to graph colouring}

The most common approach to register allocation is to reduce the problem to one of graph
colouring, where we have a graph which we wish to colour with a maximum of K colours such
that no two adjacent vertices share the same colour (TODO ref). Given the knowledge of which
variables are live at the same time, we construct a clash graph in which the vertices
represent registers and an edge between two vertices exists where the corresponding
registers are simultaneously live. Treating a colour as representing a physical register
on the target machine, we then attempt to `colour' the graph with target registers such
that vertices which are linked by a clash edge are always given different colours.

A successful colouring will naturally give a mapping from virtual to physical registers
enforcing that no registers which are simultaneously live are placed in the same physical
register. Formally verifying that the colouring resulting from this process will never
affect the behaviour of the code being compiled was the core objective of this project.

The K-colourability problem is known to be NP-complete wherever K is greater than two,
and so finding a truly optimal solution is rarely feasible in practice. Instead,
register allocation algorithms are designed primarily to find a solution which is
correct, and from there use heuristics to find a solution which is as optimal as
can be achieved in reasonable time. There are several approaches to this, and many
of them were verified in detail in this project. As discussed later, the definitions
and proofs in the project are designed to be largely independent of the algorithm
and so extending it to verify some of the algorithms not covered would be relatively
simple.

\section {Basic approaches to colouring}

The simplest way of colouring a graph is to take each vertex in turn and assign
it a new colour until all vertices have been coloured. This approach is guaranteed
to generate a valid colouring (as verified in Section (TODO section)), but is also very
inefficient as registers are never re-used for variables which could reside in the
same register. In a system with few physical registers this will result in registers
being spilled to memory, and as the order colours are assigned is arbitrary this could
result in frequently-used registers being stored in memory causing constant loads and
stores and poor performance.

An improvement is to take each vertex in turn and assign it the lowest register which
isn't already in use by a vertex with which it clashes. This is fairly simple to
implement, and significantly improves register utilisation as it re-uses old registers
whenever the opportunity presents itself (TODO ref?).

However, lowest-first allocation doesn't always find opportunities to re-use old
registers. A further refinement is to check whether we can avoid using a new register
by swapping colours of already-allocated vertices. This often improves register
utilisation further, as sub-optimal decisions made early on in the process can be
changed when colouring new vertices later.

The algorithm resulting from applying all these refinements will work fairly well,
but in practice the quality of the colouring is often also heavily dependent on the
order in which we colour vertices. Decisions made at the beginning naturally constrain
decisions made later, even if we allow the algorithm to swap existing colours. We
therefore need to use a heuristic ordering step at the beginning to improve the order
in which vertices are presented to the main colouring algorithm. Some such heuristics
are described in the next section.

\section{Heuristics}
Heuristics are used in graph-colouring algorithms to select a vertex ordering which
gives a more optimal solution. There are several heuristics which are used in practice,
some of which were formally verified during the course of this project.

One of the simplest heuristics is to order the vertices by their degree. By presenting
the vertices to the colouring algorithm in descending order of degree, vertices which
are most likely to clash with others are coloured first so that those left over towards
the end will be easier to colour without assigning new colours. There are also a few
variations on this: saturation degree ordering, for instance, picks the next vertex to
colour based on which vertex is currently adjacent to the most different colours.
Incidence degree colouring is another variant which sorts based on the number of adjacent
vertices which have already been coloured (TODO ref).

A more complex heuristic, which is useful when attempting to colour with at most K colours,
is to pick at each stage the vertex of highest degree less than K (this constraint on its
degree ensures that it can definitely be coloured if only K colours are available). This
vertex is then removed from the graph and placed on top of a vertex stack. The process
repeats, each time removing a vertex which is guaranteed to be colourable, until no
vertices of degree less than K remain. Those which have not yet been removed are spilled
to memory, and the rest of the vertices are popped off the stack one by one and passed
to the colouring algorithm. The heuristic essentially priorities registers with few
conflicts, spilling those with many conflicts to memory, and colours those of lowest
degree first.

This is not an exhaustive list of the heuristics used in practice, but these two types
are very common and so verification focussed primarily on these. As mentioned, most of
the verification conditions were designed to be as general as possible such that other
more exotic heuristics could also be verified without much additional effort.



\chapter{Related Work} 

TODO


\chapter{Live variable analysis and clash graph generation}

\section*{Introduction}

In order to achieve a complete end-to-end verification, it was decided that the verified
allocator should take in some representation of three-address code (typical of this stage
in the compilation process), and output three-address code with registers allocated. The
overall proof would therefore show that the behaviour of the resulting code was exactly
the same as that of the source in all situations. For this reason, a simplified
representation of three-address code and an evaluation function were developed and an
algorithm to generate clash graphs was verified. This chapter explores the representation
used, the way `colouring' registers was represented and the development and verification
of the clash graph generator, along with the way a `correct' register allocation was defined.
The proofs showing that these definitions are correct and linking them to the project's overall
goal will follow in Chapter (TODO: bringing everything together).

\section{Three-address code representation}

Here we explore the simplified representation of three-address code, how evaluation was
defined and how colourings were modelled and applied.

\subsection{Registers and instructions}

In real systems, register naming conventions tend to vary between processors, and there
are often multiple groups of registers available. For the purposes of verification, this
was simplified by using natural numbers to refer to registers, as in practice these numbers
could trivially be mapped to real register names. Furthermore, there will generally be a
great many different operations available on a given machine, but as the focus is on register
allocation the only thing that matters here is the registers read and those assigned.
Compilers commonly use three-address code as an intermediate representation, so a given
instruction essentially consists of two source registers and one destination register. This
led to the very simplified instruction representation given below:

\begin{alltt}\small
	\HOLTyOp{inst} = \HOLConst{Inst} \HOLKeyword{of} \HOLTyOp{num} \HOLTokenImp{} \HOLTyOp{num} \HOLTokenImp{} \HOLTyOp{num}
\end{alltt}

For the main verification task, this was entirely satisfactory as it captures all the
information needed. Later work extended this representation slightly: for instance, the
addition of preferences meant it was necessary to include move instructions so that these
could be used to compute preference graphs. Extensions to the three-address representation
are discussed in Chapter (TODO: extensions chapter).

Code blocks were then represented as simple lists of these instructions, to simplify
verification. The extension to more complex code featuring basic blocks is not too great
a leap, and is discussed in Section (TODO: extensions-complex code). The main difference
in using this representation is that live variable analysis is somewhat simpler.

\subsection{Code evaluation}

As mentioned, it was not necessary for verification to have a wide range of instructions.
The register allocation can be proven correct simply by showing that evaluating each
instruction with an arbitrary binary operator and storing the result in the destination
register has the same effect as it did before allocation. The precise meaning of `has
the same effect' is explained in Section (TODO bringing together - proving stuff about code
evaluation), where it is proven to hold for the register allocator built during this project.
The evaluation function used is given below:

\begin{alltt}\small
	\HOLConst{eval} \HOLFreeVar{f} \HOLFreeVar{s} [] = \HOLFreeVar{s}
\HOLConst{eval} \HOLFreeVar{f} \HOLFreeVar{s} (\HOLConst{Inst} \HOLFreeVar{w} \HOLFreeVar{r\sb{\mathrm{1}}} \HOLFreeVar{r\sb{\mathrm{2}}}::\HOLFreeVar{code}) =
\HOLConst{eval} \HOLFreeVar{f} ((\HOLFreeVar{w} =+ \HOLFreeVar{f} (\HOLFreeVar{s} \HOLFreeVar{r\sb{\mathrm{1}}}) (\HOLFreeVar{s} \HOLFreeVar{r\sb{\mathrm{2}}})) \HOLFreeVar{s}) \HOLFreeVar{code}
\end{alltt}

Here $f$ is an arbitrary binary operator and $s$ is the store, a total function representing
the values of all registers. Each instruction is evaluated by looking up the two source
registers in the store and applying the binary operation to them, then updating the store
function to map the destination register to the resulting value. The base case, where we
have reached the end of the code block, simply returns the store as it is.

\subsection{Colourings}

With the basic code type defined, it was then necessary to model a register allocation
abstractly in order to reason about allocations conveniently. An allocation was defined
as a `colouring', in keeping with the focus on graph-colouring approaches, mapping registers
to registers. It thus made sense simply to use a total function of type $num \rightarrow num$.
This made it very easy to later define properties of colourings -- most importantly, the notion
of a colouring not mapping conflicting virtual registers to the same target register. This
definition is discussed in Section (TODO: LVA-colouring_ok).

Applying a colouring is then very simple: the function iterates through instructions,
applying the colouring function to each register as it goes. The definition is given below:

\begin{alltt}\small
	\HOLConst{apply} \HOLFreeVar{c} [] = []
\HOLConst{apply} \HOLFreeVar{c} (\HOLConst{Inst} \HOLFreeVar{w} \HOLFreeVar{r\sb{\mathrm{1}}} \HOLFreeVar{r\sb{\mathrm{2}}}::\HOLFreeVar{code}) =
\HOLConst{Inst} (\HOLFreeVar{c} \HOLFreeVar{w}) (\HOLFreeVar{c} \HOLFreeVar{r\sb{\mathrm{1}}}) (\HOLFreeVar{c} \HOLFreeVar{r\sb{\mathrm{2}}})::\HOLConst{apply} \HOLFreeVar{c} \HOLFreeVar{code}
\end{alltt}

This completes the definition of the basic primitives used to represent code in the proofs.

\section{Live variables and clash graphs}

This section describes the definitions and functions used in producing a verified clash
graph generator, and covers the lemmas and proofs used to prove its correctness. The
first subsection demonstrates the representation of sets using lists, which was adopted
to simplify definitions and proofs, and many of the properties proved of such sets
which were vital in later proofs. Verifying a clash graph generator was then handled
in two stages: first, a live variable analysis was implemented and proven correct. A
clash graph generator was then developed using the live variable analysis, and various
important properties were verified to aid later proofs. Section (TODO: LVA-colouring_ok_def)
details how the correctness of a generated colouring was defined, a definition which was
fundamental to the rest of the project.

\subsection{Set representation and proofs}

As mentioned, it was decided that for the purposes of the project it would be most
convenient to represent sets as lists rather than using HOL's set primitives. Using
lists made it considerably easier to define functions iterating over elements of a set,
and made many of the proofs and definitions a lot easier to work with. Furthermore,
lists allow for an ordering to exist, which was particularly useful when applying
heuristics to vertices of clash graphs. The downside was that a number of properties
of sets had to be verified, and showing that various set operations preserved its
duplicate-freeness took a lot of extra effort. (The necessary proofs and definitions
related to duplicate-freeness are given in Section (TODO: colouring_ok_def - duplicate_free stuff).)

The two basic operations on sets are insertion and deletion of elements. The definitions of these for `list sets' are given below:

\begin{alltt}\small
	\HOLConst{insert} \HOLFreeVar{x} \HOLFreeVar{xs} = \HOLKeyword{if} \HOLConst{MEM} \HOLFreeVar{x} \HOLFreeVar{xs} \HOLKeyword{then} \HOLFreeVar{xs} \HOLKeyword{else} \HOLFreeVar{x}::\HOLFreeVar{xs}
	\HOLConst{delete} \HOLFreeVar{x} \HOLFreeVar{xs} = \HOLConst{FILTER} (\HOLTokenLambda{}\HOLBoundVar{y}. \HOLFreeVar{x} \HOLTokenNotEqual{} \HOLBoundVar{y}) \HOLFreeVar{xs}
\end{alltt}

Insertion simply tests for membership and places the new element at the head of the list if it is not a member, and deletion filters the list by inequality to the element to be removed. Several properties were then proven to demonstrate that these definitions work:

\begin{alltt}\small
	\HOLTokenTurnstile{} \HOLConst{MEM} \HOLFreeVar{x} (\HOLConst{insert} \HOLFreeVar{x} \HOLFreeVar{list})
\end{alltt}

This just states that inserting an element does indeed add it -- it is correct with respect to \texttt{MEM}. The proof is a simple induction.

\begin{alltt}\small
	\HOLTokenTurnstile{} \HOLConst{MEM} \HOLFreeVar{x} (\HOLConst{insert} \HOLFreeVar{y} \HOLFreeVar{list}) \HOLTokenConj{} \HOLFreeVar{x} \HOLTokenNotEqual{} \HOLFreeVar{y} \HOLTokenImp{} \HOLConst{MEM} \HOLFreeVar{x} \HOLFreeVar{list}
\end{alltt}

This states that if $x$ is a member of a set with a value inserted, and it is not equal to that value, then it must be a member of the original set. The proof is a trivial case split on whether $y$ belongs to the original set, simplifying definitions in each case to prove the goal.

\begin{alltt}\small
	\HOLTokenTurnstile{} \HOLConst{MEM} \HOLFreeVar{x} \HOLFreeVar{xs} \HOLTokenImp{} \HOLConst{MEM} \HOLFreeVar{x} (\HOLConst{insert} \HOLFreeVar{y} \HOLFreeVar{xs})
\end{alltt}

The proof of this is a simple expansion of the definition of insertion, and shows that inserting an element retains the elements already there.

\begin{alltt}\small
	\HOLTokenTurnstile{} \HOLFreeVar{a} \HOLTokenNotIn{} \HOLConst{set} (\HOLConst{insert} \HOLFreeVar{x} \HOLFreeVar{xs}) \HOLTokenImp{} \HOLFreeVar{a} \HOLTokenNotIn{} \HOLConst{set} \HOLFreeVar{xs}
\end{alltt}

The contrapositive of the above, this is a useful way of simplifying goals where some value isn't an element of a generated set.

\begin{alltt}\small
	\HOLTokenTurnstile{} \HOLFreeVar{x} \HOLTokenNotIn{} \HOLConst{set} (\HOLConst{delete} \HOLFreeVar{x} \HOLFreeVar{list})
\end{alltt}

The basic proof of delete's correctness, this shows that deleting a value from a set does in fact remove it. The proof follows by expanding the definitions of \texttt{delete} and \texttt{filter}.

It was also necessary to implement set union, and union over multiple sets. The definition of set union is given below:

\begin{alltt}\small
	\HOLConst{list_union} [] \HOLFreeVar{ys} = \HOLFreeVar{ys}
\HOLConst{list_union} (\HOLFreeVar{x}::\HOLFreeVar{xs}) \HOLFreeVar{ys} = \HOLConst{insert} \HOLFreeVar{x} (\HOLConst{list_union} \HOLFreeVar{xs} \HOLFreeVar{ys})
\end{alltt}

It was necessary to show that this definitely contains all elements from both sets. The relevant property is given below, and was proven by induction on the size of the first set in the union by demonstrating that once $x$ is reached it is guaranteed to be included in the result.

\begin{alltt}\small
	\HOLTokenTurnstile{} \HOLConst{MEM} \HOLFreeVar{x} \HOLFreeVar{list} \HOLTokenDisj{} \HOLConst{MEM} \HOLFreeVar{x} \HOLFreeVar{list\sp{\prime}} \HOLTokenImp{} \HOLConst{MEM} \HOLFreeVar{x} (\HOLConst{list_union} \HOLFreeVar{list} \HOLFreeVar{list\sp{\prime}})
\end{alltt}

Union over multiple sets, here referred to as a flattening union, is defined by recursively applying \texttt{list_union} to pairs of sets from the end of the list forwards:

\begin{alltt}\small
	\HOLConst{list_union_flatten} [] = []
\HOLConst{list_union_flatten} (\HOLFreeVar{l}::\HOLFreeVar{ls}) =
\HOLConst{list_union} \HOLFreeVar{l} (\HOLConst{list_union_flatten} \HOLFreeVar{ls})
\end{alltt}

Again, a completeness proof was required for later proofs. This is given below:

\begin{alltt}\small
	\HOLTokenTurnstile{} \HOLConst{MEM} \HOLFreeVar{list} \HOLFreeVar{lists} \HOLTokenConj{} \HOLConst{MEM} \HOLFreeVar{x} \HOLFreeVar{list} \HOLTokenImp{}
   \HOLConst{MEM} \HOLFreeVar{x} (\HOLConst{list_union_flatten} \HOLFreeVar{lists})
\end{alltt}

The proof used the same technique as the last one, combined with the fact that the union of two lists contains all elements from both lists (as was just proved).

This completes the basic definitions and proofs for list-based sets. An intersection operation was not required for the sets used, as there was no function in the algorithm which used it. Additionally, only completeness of unions was required -- interestingly, it was not necessary to prove that they didn't introduce any extra elements for any later proofs. This is a side-effect of all live variable analysis methods generally computing safe over-approximations: an algorithm which satisfies an overly large set of conflicts will automatically satisfy a smaller, more precise set.

\subsection{Live variable analysis}

As mentioned earlier, the decision to represent a block of code as a simple list of instructions makes live variable analysis noticeably less complex. The definition used is given below:

\begin{alltt}\small
	\HOLConst{get_live} [] \HOLFreeVar{live} = \HOLFreeVar{live}
\HOLConst{get_live} (\HOLConst{Inst} \HOLFreeVar{w} \HOLFreeVar{r\sb{\mathrm{1}}} \HOLFreeVar{r\sb{\mathrm{2}}}::\HOLFreeVar{code}) \HOLFreeVar{live} =
\HOLConst{insert} \HOLFreeVar{r\sb{\mathrm{1}}} (\HOLConst{insert} \HOLFreeVar{r\sb{\mathrm{2}}} (\HOLConst{delete} \HOLFreeVar{w} (\HOLConst{get_live} \HOLFreeVar{code} \HOLFreeVar{live})))
\end{alltt}

This is an implementation of the liveness equation given in Section (TODO background bit), which also allows for some variables to be live at the end of the program (which can be used to link together analyses for adjacent blocks of code, or specify which registers hold program results). Showing that this definition is correct is subtle: correctness with respect to register allocation is shown later by demonstrating that colourings satisfying constraints built up from this definition will not change code behaviour. However, another way of showing correctness is by proving that only the variables returned by \texttt{get_live} will affect the evaluation of a program, a property which is stated below:

\begin{alltt}\small
	\HOLTokenTurnstile{} (\HOLConst{MAP} \HOLFreeVar{s} (\HOLConst{get_live} \HOLFreeVar{code} \HOLFreeVar{live}) = \HOLConst{MAP} \HOLFreeVar{t} (\HOLConst{get_live} \HOLFreeVar{code} \HOLFreeVar{live})) \HOLTokenImp{}
   (\HOLConst{MAP} (\HOLConst{eval} \HOLFreeVar{f} \HOLFreeVar{s} \HOLFreeVar{code}) \HOLFreeVar{live} = \HOLConst{MAP} (\HOLConst{eval} \HOLFreeVar{f} \HOLFreeVar{t} \HOLFreeVar{code}) \HOLFreeVar{live})
\end{alltt}

The antecedent states that the two stores $s$ and $t$ agree on the variables which are live at the start of the code. The consequent then claims that the stores at the end of the program's evaluation will agree on whatever variables are designated as live at the end. The proof of this statement is fairly complex, and is (TODO write up proof).

\subsection{Defining correctness of a colouring}

In order to be able to prove that generated colourings are correct, and that correct colouring do not affect code behaviour, defining the concept of a `correct' colouring was essential. Fundamentally, a valid colouring is one which will not map two simultaneously live variables to the same target register. That is to say, for the set of variables live at any given instruction, mapping the colouring over them will not result in any duplicates. To facilitate this sort of definition, the property of being duplicate-free was defined as a function and a number of properties were proven of this. These are outlined in the next subsection.

\subsubsection{Duplicate-freeness and associated lemmas}

Duplicate-freeness of a list was defined using the following function:

\begin{alltt}\small
	\HOLConst{duplicate_free} [] \HOLTokenEquiv{} \HOLConst{T}
\HOLConst{duplicate_free} (\HOLFreeVar{x}::\HOLFreeVar{xs}) \HOLTokenEquiv{} \HOLFreeVar{x} \HOLTokenNotIn{} \HOLConst{set} \HOLFreeVar{xs} \HOLTokenConj{} \HOLConst{duplicate_free} \HOLFreeVar{xs}
\end{alltt}

To open up more options for proving duplicate-freeness, a pair of lemmas describing its behaviour in terms of element equality were also proved. The first shows that if no two different elements of the list are equal, it is duplicate-free:

\begin{alltt}\small
	\HOLTokenTurnstile{} (\HOLTokenForall{}\HOLBoundVar{x} \HOLBoundVar{y}.
      \HOLBoundVar{x} \HOLTokenLt{} \HOLConst{LENGTH} \HOLFreeVar{list} \HOLTokenConj{} \HOLBoundVar{y} \HOLTokenLt{} \HOLConst{LENGTH} \HOLFreeVar{list} \HOLTokenConj{} \HOLBoundVar{x} \HOLTokenNotEqual{} \HOLBoundVar{y} \HOLTokenImp{}
      \HOLConst{EL} \HOLBoundVar{x} \HOLFreeVar{list} \HOLTokenNotEqual{} \HOLConst{EL} \HOLBoundVar{y} \HOLFreeVar{list}) \HOLTokenImp{}
   \HOLConst{duplicate_free} \HOLFreeVar{list}
\end{alltt}

The second comes from the other side, stating that if a list is duplicate-free, no two elements will be equal:

\begin{alltt}\small
	\HOLTokenTurnstile{} \HOLConst{duplicate_free} \HOLFreeVar{list} \HOLTokenConj{} \HOLFreeVar{x} \HOLTokenLt{} \HOLConst{LENGTH} \HOLFreeVar{list} \HOLTokenConj{} \HOLFreeVar{y} \HOLTokenLt{} \HOLConst{LENGTH} \HOLFreeVar{list} \HOLTokenConj{}
   \HOLFreeVar{x} \HOLTokenNotEqual{} \HOLFreeVar{y} \HOLTokenImp{}
   \HOLConst{EL} \HOLFreeVar{x} \HOLFreeVar{list} \HOLTokenNotEqual{} \HOLConst{EL} \HOLFreeVar{y} \HOLFreeVar{list}
\end{alltt}

The proof was fairly long as there were several details to account for, but essentially boiled down to an induction where the inductive step case splits on $x$ or $y$ being equal to zero. If neither is, the inductive hypothesis proves the goal; otherwise, the other must be greater than zero and the definition of duplicate-freeness shows that the value at zero cannot be a member of the rest of the list.

It was then necessary to prove that the set operations defined on list sets preserve duplicate-freeness, as required. The statements for insertion and deletion follow:

\begin{alltt}\small
	\HOLTokenTurnstile{} \HOLConst{duplicate_free} (\HOLConst{insert} \HOLFreeVar{n} \HOLFreeVar{list}) \HOLTokenEquiv{} \HOLConst{duplicate_free} \HOLFreeVar{list}
	\HOLTokenTurnstile{} \HOLConst{duplicate_free} (\HOLConst{delete} \HOLFreeVar{n} \HOLFreeVar{list}) \HOLTokenEquiv{} \HOLConst{duplicate_free} \HOLFreeVar{list}
\end{alltt}

The first follows from the definitions, and the second was shown by induction.

Duplicate-freeness was then proven for unions of duplicate-free sets, first by proving that it holds for a simple union of two list sets:

\begin{alltt}\small
	\HOLTokenTurnstile{} \HOLConst{duplicate_free} \HOLFreeVar{xs} \HOLTokenConj{} \HOLConst{duplicate_free} \HOLFreeVar{ys} \HOLTokenImp{}
   \HOLConst{duplicate_free} (\HOLConst{list_union} \HOLFreeVar{xs} \HOLFreeVar{ys})
\end{alltt}

This was shown by induction, using the fact that unions rely on insertion which preserves duplicate-freeness.

This was then extended to general unions. The proof works by induction, using the above lemma and expanding definitions.

\begin{alltt}\small
	\HOLTokenTurnstile{} \HOLConst{EVERY} (\HOLTokenLambda{}\HOLBoundVar{list}. \HOLConst{duplicate_free} \HOLBoundVar{list}) \HOLFreeVar{lists} \HOLTokenImp{}
   \HOLConst{duplicate_free} (\HOLConst{list_union_flatten} \HOLFreeVar{lists})
\end{alltt}

The next objective was to show that the set of registers returned by get_live contains no duplicates, assuming there are no duplicates in the set of registers considered to be live at the end of the code. This was a fairly simple proof by induction, using the lemmas duplicate_free_insertion and duplicate_free_deletion and the definition of get_live:

\begin{alltt}\small
	\HOLTokenTurnstile{} \HOLConst{duplicate_free} \HOLFreeVar{live} \HOLTokenImp{} \HOLConst{duplicate_free} (\HOLConst{get_live} \HOLFreeVar{code} \HOLFreeVar{live})
\end{alltt}

Finally, the following useful property was proved:

\begin{alltt}\small
	\HOLTokenTurnstile{} \HOLConst{duplicate_free} (\HOLConst{MAP} \HOLFreeVar{c} \HOLFreeVar{live}) \HOLTokenConj{} \HOLFreeVar{x} \HOLTokenNotEqual{} \HOLFreeVar{y} \HOLTokenConj{} \HOLConst{MEM} \HOLFreeVar{x} \HOLFreeVar{live} \HOLTokenConj{}
   \HOLConst{MEM} \HOLFreeVar{y} \HOLFreeVar{live} \HOLTokenImp{}
   \HOLFreeVar{c} \HOLFreeVar{x} \HOLTokenNotEqual{} \HOLFreeVar{c} \HOLFreeVar{y}
\end{alltt}

This states that if the result of mapping a function over a list is duplicate-free, and two elements of the original list are not equal, then the function maps them to different things. This lemma was particularly useful in proving injectivity properties of valid colourings later. The proof is by induction on the list $live$, where in the inductive case we expand the definitions and case split on whether $x$ or $y$ is equal to the current head of the list. In either case, the other must be behind in the list and inequality of $c x$ and $c y$ follows by the definition of duplicate-freeness.

\subsubsection{Basic definition of colouring correctness}

As discussed earlier, a colouring is deemed `correct' if for any instruction's set of live variables, the result of mapping the colouring function over it is duplicate-free. This is relatively easy to define:

\begin{alltt}\small
	\HOLConst{colouring_ok} \HOLFreeVar{c} [] \HOLFreeVar{live} \HOLTokenEquiv{} \HOLConst{duplicate_free} (\HOLConst{MAP} \HOLFreeVar{c} \HOLFreeVar{live})
\HOLConst{colouring_ok} \HOLFreeVar{c} (\HOLConst{Inst} \HOLFreeVar{w} \HOLFreeVar{r\sb{\mathrm{1}}} \HOLFreeVar{r\sb{\mathrm{2}}}::\HOLFreeVar{code}) \HOLFreeVar{live} \HOLTokenEquiv{}
\HOLConst{duplicate_free} (\HOLConst{MAP} \HOLFreeVar{c} (\HOLConst{get_live} (\HOLConst{Inst} \HOLFreeVar{w} \HOLFreeVar{r\sb{\mathrm{1}}} \HOLFreeVar{r\sb{\mathrm{2}}}::\HOLFreeVar{code}) \HOLFreeVar{live})) \HOLTokenConj{}
\HOLConst{colouring_ok} \HOLFreeVar{c} \HOLFreeVar{code} \HOLFreeVar{live}
\end{alltt}

An important property of colouring correctness is that it is also correct for a subset of the code block, and this statement is given below. The proof of this is a trivial expansion of the definition of \texttt{colouring\_ok}.

\begin{alltt}\small
	\HOLTokenTurnstile{} \HOLConst{colouring_ok} \HOLFreeVar{c} (\HOLConst{Inst} \HOLFreeVar{n} \HOLFreeVar{n\sb{\mathrm{0}}} \HOLFreeVar{n\sb{\mathrm{1}}}::\HOLFreeVar{code}) \HOLFreeVar{live} \HOLTokenImp{}
   \HOLConst{colouring_ok} \HOLFreeVar{c} \HOLFreeVar{code} \HOLFreeVar{live}
\end{alltt}

This definition works nicely for the proofs involving working directly with three-address code, and was used for most of the later proofs regarding code evaluation behaviour. However, it is very strongly tied to the code definitions and doesn't clearly reflect the conflicts graph colouring is working with, making proofs relating graph colourings to valid register allocations particularly difficult.

For this reason, an alternative definition of \texttt{colouring\_ok} was created and proved to be equivalent to the original definition. This alternative definition was much easier to work with when reasoning about clash graphs, as demonstrated later in Section (TODO bringing together - satisfactory implies ok).

\subsubsection{Alternative \texttt{colouring\_ok} definition}

The alternative colouring_ok definition was designed to be slightly more abstract than the original one: rather than directly working with \texttt{get_live}, it uses a function \texttt{colouring\_respects\_conflicting\_sets} to determine whether a colouring respects an arbitrary list of conflicting sets. Here we say a colouring respects a conflicting set if mapping it over the elements of the set produces a duplicate-free result. The definition of this function is given below:

\begin{alltt}\small
	\HOLConst{colouring_respects_conflicting_sets} \HOLFreeVar{c} [] \HOLTokenEquiv{} \HOLConst{T}
\HOLConst{colouring_respects_conflicting_sets} \HOLFreeVar{c} (\HOLFreeVar{s}::\HOLFreeVar{ss}) \HOLTokenEquiv{}
\HOLConst{duplicate_free} (\HOLConst{MAP} \HOLFreeVar{c} \HOLFreeVar{s}) \HOLTokenConj{}
\HOLConst{colouring_respects_conflicting_sets} \HOLFreeVar{c} \HOLFreeVar{ss}
\end{alltt}

This can also be conveniently expressed using HOL's \texttt{EVERY} function, as shown in the following theorem:

\begin{alltt}\small
	\HOLTokenTurnstile{} \HOLConst{colouring_respects_conflicting_sets} \HOLFreeVar{c} \HOLFreeVar{sets} \HOLTokenEquiv{}
   \HOLConst{EVERY} (\HOLTokenLambda{}\HOLBoundVar{list}. \HOLConst{duplicate_free} (\HOLConst{MAP} \HOLFreeVar{c} \HOLBoundVar{list})) \HOLFreeVar{sets}
\end{alltt}

The proof is a trivial induction on the number of conflicting sets where each case is solved by expanding definitions and using \texttt{METIS_TAC}.

Clearly the conflicting sets to use in this case are just the sets of live variables at each instruction. To keep things abstract, this was pulled into a separate function, defined as shown below:

\begin{alltt}\small
	\HOLConst{conflicting_sets} [] \HOLFreeVar{live} = [\HOLFreeVar{live}]
\HOLConst{conflicting_sets} (\HOLConst{Inst} \HOLFreeVar{w} \HOLFreeVar{r\sb{\mathrm{1}}} \HOLFreeVar{r\sb{\mathrm{2}}}::\HOLFreeVar{code}) \HOLFreeVar{live} =
\HOLConst{get_live} (\HOLConst{Inst} \HOLFreeVar{w} \HOLFreeVar{r\sb{\mathrm{1}}} \HOLFreeVar{r\sb{\mathrm{2}}}::\HOLFreeVar{code}) \HOLFreeVar{live}::\HOLConst{conflicting_sets} \HOLFreeVar{code} \HOLFreeVar{live}
\end{alltt}

With this in hand, \texttt{colouring\_ok\_alt} can be defined by passing the results of this function into \texttt{colouring\_respects\_conflicting\_sets}:

\begin{alltt}\small
	\HOLConst{colouring_ok_alt} \HOLFreeVar{c} \HOLFreeVar{code} \HOLFreeVar{live} \HOLTokenEquiv{}
\HOLConst{colouring_respects_conflicting_sets} \HOLFreeVar{c}
  (\HOLConst{conflicting_sets} \HOLFreeVar{code} \HOLFreeVar{live})
\end{alltt}

The advantage of abstracting details in this way is that we can prove properties of \texttt{conflicting\_sets} and \texttt{colouring\_respects\_conflicting\_sets} which allow us to avoid working directly with instructions altogether. This makes it easier to make changes to the way code is represented, as is done for some of the extensions in Chapter (TODO extensions).

For these two definitions to be used interchangeably in proofs, it was necessary to prove that they were equivalent:

\begin{alltt}\small
	\HOLTokenTurnstile{} \HOLConst{colouring_ok_alt} \HOLFreeVar{c} \HOLFreeVar{code} \HOLFreeVar{live} \HOLTokenEquiv{} \HOLConst{colouring_ok} \HOLFreeVar{c} \HOLFreeVar{code} \HOLFreeVar{live}
\end{alltt}

The proof is a trivial induction where the inductive case is shown by simplification.

\subsection{Clash graph generation}

With the verified live variable analysis complete, the next step was to use this to generate clash graphs which could be fed to the graph colouring algorithms, and to verify that these graphs accurately reflected the register conflicts present in the code.

Initial work here focussed on finding the graph representation which would make the definitions and proofs easiest to work with. A number of alternatives were considered, but it was ultimately decided that the best option would be a set of $(v, es)$ pairs where $v$ is a vertex and $es$ is the set of vertices with which $v$ shares an edge. This works best because it allows vertices to be considered in any order and lets the code quickly look up which vertices it conflicts with without having to traverse the graph. Representing conflicts between vertices using a HOL relation (that is, a function of type $num \times num \rightarrow bool$) was considered, but this made it difficult to iterate over conflicting registers and was more awkward to construct given the results of live variable analysis.

\subsubsection{Computing conflicts for a register}

The first step in generating such a graph was to generate a list set of all the conflicts for a particular register. This used the following function:

\begin{alltt}\small
	\HOLConst{conflicts_for_register} \HOLFreeVar{r} \HOLFreeVar{code} \HOLFreeVar{live} =
\HOLConst{delete} \HOLFreeVar{r}
  (\HOLConst{list_union_flatten}
     (\HOLConst{FILTER} (\HOLTokenLambda{}\HOLBoundVar{set}. \HOLConst{MEM} \HOLFreeVar{r} \HOLBoundVar{set}) (\HOLConst{conflicting_sets} \HOLFreeVar{code} \HOLFreeVar{live})))
\end{alltt}

The definition is slightly complex. The function first finds the list of conflicting sets for the code using the \texttt{conflicting\_sets} function defined in the previous subsection. It then filters these down to only those sets which the current register is a member of, and applies \texttt{list\_union\_flatten} to combine these into a single set of all registers belonging to a conflicting set which the current register also belongs to. Finally, it removes the current register, as a register should not appear in its own list of conflicts, and returns the result.

The result is a duplicate-free set list containing all the registers a given register conflicts with. Its duplicate-freeness is verified below, and was essential in later proofs:

\begin{alltt}\small
	\HOLTokenTurnstile{} \HOLConst{duplicate_free} \HOLFreeVar{live} \HOLTokenImp{}
   \HOLConst{duplicate_free} (\HOLConst{conflicts_for_register} \HOLFreeVar{r} \HOLFreeVar{code} \HOLFreeVar{live})
\end{alltt}

To prove this we need the following property of conflicting sets, which is a simple induction using the fact that \texttt{get\_live} is duplicate-free:

\begin{alltt}\small
	\HOLTokenTurnstile{} \HOLConst{duplicate_free} \HOLFreeVar{live} \HOLTokenImp{}
   \HOLConst{EVERY} (\HOLTokenLambda{}\HOLBoundVar{list}. \HOLConst{duplicate_free} \HOLBoundVar{list})
     (\HOLConst{conflicting_sets} \HOLFreeVar{code} \HOLFreeVar{live})
\end{alltt}

We can then show that filtering this set yields a duplicate-free set using the following property of \texttt{EVERY}, verified by trivial induction:

\begin{alltt}\small
	\HOLTokenTurnstile{} \HOLConst{EVERY} \HOLFreeVar{P} \HOLFreeVar{list} \HOLTokenImp{} \HOLConst{EVERY} \HOLFreeVar{P} (\HOLConst{FILTER} \HOLFreeVar{Q} \HOLFreeVar{list})
\end{alltt}

The goal then follows as the flattening union function and deletion both preserve duplicate-freeness.

\subsubsection{Constructing the graph}

Once the function to determine a register's conflicts had been defined, the function to construct the whole graph was fairly simple to construct:

\begin{alltt}\small
	\HOLConst{get_conflicts} \HOLFreeVar{code} \HOLFreeVar{live} =
\HOLConst{MAP} (\HOLTokenLambda{}\HOLBoundVar{reg}. (\HOLBoundVar{reg},\HOLConst{conflicts_for_register} \HOLBoundVar{reg} \HOLFreeVar{code} \HOLFreeVar{live}))
  (\HOLConst{get_registers} \HOLFreeVar{code} \HOLFreeVar{live})
\end{alltt}

This just maps the \texttt{conflicts\_for\_register} function over all registers used in the program, putting the results into a tuple with the number of the register. The result is a graph of the following form:

$[ (r_1, [c_1, \ldots, c_n]), \ldots, (r_n, [c_1, \ldots, c_n]) ]$

Here $r_n$ is the $n^{th}$ register and $c_n$ is the $n^{th}$ conflicting register. The definition depends on a supporting function which finds the set of all registers used by a block of code:

\begin{alltt}\small
	\HOLConst{get_registers} [] \HOLFreeVar{live} = \HOLFreeVar{live}
\HOLConst{get_registers} (\HOLConst{Inst} \HOLFreeVar{r\sb{\mathrm{0}}} \HOLFreeVar{r\sb{\mathrm{1}}} \HOLFreeVar{r\sb{\mathrm{2}}}::\HOLFreeVar{code}) \HOLFreeVar{live} =
\HOLConst{insert} \HOLFreeVar{r\sb{\mathrm{0}}} (\HOLConst{insert} \HOLFreeVar{r\sb{\mathrm{1}}} (\HOLConst{insert} \HOLFreeVar{r\sb{\mathrm{2}}} (\HOLConst{get_registers} \HOLFreeVar{code} \HOLFreeVar{live})))
\end{alltt}

Note that this is trivially duplicate-free where $live$ is duplicate-free, because it only uses set insertions and these have been proven to preserve duplicate-freeness:

\begin{alltt}\small
	\HOLTokenTurnstile{} \HOLConst{duplicate_free} \HOLFreeVar{live} \HOLTokenImp{}
   \HOLConst{duplicate_free} (\HOLConst{get_registers} \HOLFreeVar{code} \HOLFreeVar{live})
\end{alltt}

\subsubsection{Proof of correctness}

This section contains the necessary proofs demonstrating that the clash graph generator works as it should. Three properties were shown: registers not included in the results of \texttt{get\_registers} do not feature in any program instruction, in \texttt{get\_live} or in any conflicting set (and so no vertices are missing from the graph), registers do not conflict with themselves, and any pair of registers in the same conflicting set will appear in each other's list of conflicts. These theorems are revisited later in Section (TODO bringing it all together), where they are used to help connect the graph colouring proofs to the proofs about code and prove the overall goal of the project.

The first thing to prove was that registers not included in \texttt{get\_registers} are never used in any program instruction:

\begin{alltt}\small
	\HOLTokenTurnstile{} \HOLFreeVar{r} \HOLTokenNotIn{} \HOLConst{set} (\HOLConst{get_registers} \HOLFreeVar{code} \HOLFreeVar{live}) \HOLTokenConj{}
   \HOLConst{MEM} (\HOLConst{Inst} \HOLFreeVar{w} \HOLFreeVar{r\sb{\mathrm{1}}} \HOLFreeVar{r\sb{\mathrm{2}}}) \HOLFreeVar{code} \HOLTokenImp{}
   \HOLFreeVar{r} \HOLTokenNotEqual{} \HOLFreeVar{w} \HOLTokenConj{} \HOLFreeVar{r} \HOLTokenNotEqual{} \HOLFreeVar{r\sb{\mathrm{1}}} \HOLTokenConj{} \HOLFreeVar{r} \HOLTokenNotEqual{} \HOLFreeVar{r\sb{\mathrm{2}}}
\end{alltt}

The proof of this statement is by induction and expanding definitions.

It follows that an unused register will never feature in the results of \texttt{get\_live}, as \texttt{get\_live} only inserts registers which are used in instructions:

\begin{alltt}\small
	\HOLTokenTurnstile{} \HOLFreeVar{r} \HOLTokenNotIn{} \HOLConst{set} (\HOLConst{get_registers} \HOLFreeVar{code} \HOLFreeVar{live}) \HOLTokenImp{}
   \HOLFreeVar{r} \HOLTokenNotIn{} \HOLConst{set} (\HOLConst{get_live} \HOLFreeVar{code} \HOLFreeVar{live})
\end{alltt}

Using the above two theorems, it was possible to prove that an unused register does not feature in any conflicting set by induction on the code:

\begin{alltt}\small
	\HOLTokenTurnstile{} \HOLFreeVar{r} \HOLTokenNotIn{} \HOLConst{set} (\HOLConst{get_registers} \HOLFreeVar{code} \HOLFreeVar{live}) \HOLTokenConj{}
   \HOLConst{MEM} \HOLFreeVar{set} (\HOLConst{conflicting_sets} \HOLFreeVar{code} \HOLFreeVar{live}) \HOLTokenImp{}
   \HOLFreeVar{r} \HOLTokenNotIn{} \HOLConst{set} \HOLFreeVar{set}
\end{alltt}

A consequence of the last theorem is that evaluating \texttt{get\_conflicts} on an unused register will yield the empty list, a property which allows us to specify definitions over all registers without caring about those which don't get used in the code. This useful property is given below:

\begin{alltt}\small
	\HOLTokenTurnstile{} \HOLFreeVar{r} \HOLTokenNotIn{} \HOLConst{set} (\HOLConst{get_registers} \HOLFreeVar{code} \HOLFreeVar{live}) \HOLTokenImp{}
   (\HOLConst{conflicts_for_register} \HOLFreeVar{r} \HOLFreeVar{code} \HOLFreeVar{live} = [])
\end{alltt}


Having effectively shown that \texttt{get\_registers} covers all the registers that need to be handled, the next stage in verifying the clash graph generator was to show that a register never appears in its own list of conflicts. This proof follows easily from the definition of \texttt{conflicts\_for\_register} as the function removes the register at the end:

\begin{alltt}\small
	\HOLTokenTurnstile{} \HOLFreeVar{r} \HOLTokenNotIn{} \HOLConst{set} (\HOLConst{conflicts_for_register} \HOLFreeVar{r} \HOLFreeVar{code} \HOLFreeVar{live})
\end{alltt}

The proof uses the following simple lemma, proved by induction on the list:

\begin{alltt}\small
	\HOLTokenTurnstile{} \HOLFreeVar{x} \HOLTokenNotIn{} \HOLConst{set} (\HOLConst{FILTER} (\HOLTokenLambda{}\HOLBoundVar{y}. \HOLFreeVar{x} \HOLTokenNotEqual{} \HOLBoundVar{y}) \HOLFreeVar{list})
\end{alltt}


To complete the correctness proof, we now verify that generated clash graphs are complete in that any two registers in the same conflicting set will appear in each other's list of conflicts:

\begin{alltt}\small
	\HOLTokenTurnstile{} \HOLConst{MEM} \HOLFreeVar{c} (\HOLConst{conflicting_sets} \HOLFreeVar{code} \HOLFreeVar{live}) \HOLTokenConj{} \HOLConst{MEM} \HOLFreeVar{r} \HOLFreeVar{c} \HOLTokenConj{} \HOLConst{MEM} \HOLFreeVar{s} \HOLFreeVar{c} \HOLTokenConj{}
   \HOLFreeVar{r} \HOLTokenNotEqual{} \HOLFreeVar{s} \HOLTokenImp{}
   \HOLConst{MEM} \HOLFreeVar{r} (\HOLConst{conflicts_for_register} \HOLFreeVar{s} \HOLFreeVar{code} \HOLFreeVar{live})
\end{alltt}

The proof depends on the following lemma, which states that if a list of lists is being filtered for whether they contain $x$, and $x$ is in $list$, $list$ is in the result:

\begin{alltt}\small
	\HOLTokenTurnstile{} \HOLConst{MEM} \HOLFreeVar{list} \HOLFreeVar{lists} \HOLTokenConj{} \HOLConst{MEM} \HOLFreeVar{x} \HOLFreeVar{list} \HOLTokenImp{}
   \HOLConst{MEM} \HOLFreeVar{list} (\HOLConst{FILTER} (\HOLTokenLambda{}\HOLBoundVar{list}. \HOLConst{MEM} \HOLFreeVar{x} \HOLBoundVar{list}) \HOLFreeVar{lists})
\end{alltt}

This can be shown by induction, case splitting on whether the current list head is equal to the list containing $x$. The goal above now follows by using this lemma to show that both registers appear in the result of filtering conflicting sets by membership, and by simplifying definitions they can be shown to appear in the overall result of \texttt{conflicts\_for\_register}.

This concludes the proof of correctness of the clash graph generator. These theorems will be used later to help prove the overall correctness of the register allocator in Section (TODO bringing it all together).


\chapter{Colouring algorithms}

\section*{Introduction}

The core verification task of this project is, of course, to prove the correctness of the algorithms that perform the register allocation. In this chapter the focus is on the algorithms and heuristics used to colour clash graphs, how correctness was modelled and how it was verified for the algorithms used. In the interests of having a minimal algorithm verified early on, a simple colouring algorithm which simply assigned each vertex its own colour was verified first. The next objective was to verify an algorithm which re-uses old colours wherever possible by assigning to each vertex the lowest register which doesn't conflict with its neighbours. From there, the work focussed on extra heuristics which could improve the performance of the lowest-first greedy algorithm by considering vertices in a different order.

\section{Modelling colouring correctness}

\subsection{Defining correctness}

Much as correctness of colouring was defined in terms of code with \texttt{colouring\_ok} and \texttt{colouring\_ok\_alt}, correctness with respect to a clash graph was defined through the function \texttt{colouring\_satisfactory}\footnote{The proof that this is equivalent follows in Section (TODO bringing it all together)}:

\begin{alltt}\small
	\HOLConst{colouring_satisfactory} \HOLFreeVar{col} [] \HOLTokenEquiv{} \HOLConst{T}
\HOLConst{colouring_satisfactory} \HOLFreeVar{col} ((\HOLFreeVar{r},\HOLFreeVar{rs})::\HOLFreeVar{cs}) \HOLTokenEquiv{}
\HOLFreeVar{col} \HOLFreeVar{r} \HOLTokenNotIn{} \HOLConst{set} (\HOLConst{MAP} \HOLFreeVar{col} \HOLFreeVar{rs}) \HOLTokenConj{} \HOLConst{colouring_satisfactory} \HOLFreeVar{col} \HOLFreeVar{cs}
\end{alltt}

This simply states that colouring a vertex and all its neighbours will not lead to it having the same colour as any of its neighbours, as required for a correct colouring. It thus breaks the proof that a given colouring is correct into two steps: it needs to respect the set of conflicts for the current vertex, and it needs to work for the rest of the vertices of the graph. As the colouring algorithms colour one vertex at a time, the second part of the proof should intuitively be correct as the current set of conflicts represent all the constraints on the current vertex. However, it was fairly difficult to verify in practice, as discussed in some of the later sections in this chapter.

\subsection{Requirements on clash graphs}

For colouring algorithms to work properly, the clash graphs need to be \textit{well-formed} in that every edge list is well-formed according to the following definition:

\begin{alltt}\small
	\HOLConst{edge_list_well_formed} (\HOLFreeVar{v},\HOLFreeVar{edges}) \HOLTokenEquiv{}
\HOLFreeVar{v} \HOLTokenNotIn{} \HOLConst{set} \HOLFreeVar{edges} \HOLTokenConj{} \HOLConst{duplicate_free} \HOLFreeVar{edges}
\end{alltt}

This specifies that a vertex cannot be linked to itself, and the set of vertices it is adjacent to cannot have any duplicates. Overall well-formedness is then given by the following definition:

\begin{alltt}\small
	\HOLConst{graph_edge_lists_well_formed} \HOLFreeVar{es} \HOLTokenEquiv{}
\HOLConst{EVERY} (\HOLTokenLambda{}\HOLBoundVar{x}. \HOLConst{edge_list_well_formed} \HOLBoundVar{x}) \HOLFreeVar{es}
\end{alltt}

In Section (TODO bringing it all together - generated graphs have required properties), the proof that graphs generated by get_conflicts have these required properties is described, but for now it is assumed that they will hold of any graph passed in to the colouring algorithms.

\section{Naive graph colouring}

A very naive graph colouring algorithm was verified first to obtain a complete proof as early into the project as possible which could be built upon later. This section describes the definition and verification of this simple colouring approach.

\subsection{The algorithm}

The naive algorithm simply allocates a colour for every register and assigns them in ascending order. This uses an auxiliary function, \texttt{naive\_colouring\_aux}:

\begin{alltt}\small
	\HOLConst{naive_colouring_aux} [] \HOLFreeVar{n} = (\HOLTokenLambda{}\HOLBoundVar{x}. \HOLFreeVar{n})
\HOLConst{naive_colouring_aux} ((\HOLFreeVar{r},\HOLFreeVar{rs})::\HOLFreeVar{cs}) \HOLFreeVar{n} =
(\HOLFreeVar{r} =+ \HOLFreeVar{n}) (\HOLConst{naive_colouring_aux} \HOLFreeVar{cs} (\HOLFreeVar{n} + 1))
\end{alltt}

This algorithm assigns registers from $n$ upwards. The main naive colouring function simply calls this starting from zero:

\begin{alltt}\small
	\HOLConst{naive_colouring} \HOLFreeVar{constraints} = \HOLConst{naive_colouring_aux} \HOLFreeVar{constraints} 0
\end{alltt}

The result is a very simple colouring method which is intuitively correct. However, verifying its correctness was slightly harder than expected, relying on several lemmas about \texttt{naive\_colouring\_aux}.

\subsection{Correctness proof}

Correctness of the algorithm was proved through correctness of \texttt{naive\_colouring\_aux}. The two statements of correctness follow:

\begin{alltt}\small
	\HOLTokenTurnstile{} \HOLConst{graph_edge_lists_well_formed} \HOLFreeVar{cs} \HOLTokenImp{}
   \HOLTokenForall{}\HOLBoundVar{n}. \HOLConst{colouring_satisfactory} (\HOLConst{naive_colouring_aux} \HOLFreeVar{cs} \HOLBoundVar{n}) \HOLFreeVar{cs}
	\HOLTokenTurnstile{} \HOLConst{graph_edge_lists_well_formed} \HOLFreeVar{cs} \HOLTokenImp{}
   \HOLConst{colouring_satisfactory} (\HOLConst{naive_colouring} \HOLFreeVar{cs}) \HOLFreeVar{cs}
\end{alltt}

The second clearly follows from the first by expanding definitions and instantiating variables:

\begin{alltt}\small
	\HOLTokenTurnstile{} (\HOLTokenForall{}\HOLBoundVar{n}. \HOLConst{colouring_satisfactory} (\HOLConst{naive_colouring_aux} \HOLFreeVar{cs} \HOLBoundVar{n}) \HOLFreeVar{cs}) \HOLTokenImp{}
   \HOLConst{colouring_satisfactory} (\HOLConst{naive_colouring} \HOLFreeVar{cs}) \HOLFreeVar{cs}
\end{alltt}

Thus we need only prove correctness of \texttt{naive\_colouring\_aux}. This required several lemmas describing the behaviour of the function and related definitions.

The first main goal was to show that each colour assigned by the algorithm has not been used before:

\begin{alltt}\small
	\HOLTokenTurnstile{} \HOLConst{naive_colouring_aux} \HOLFreeVar{cs} (\HOLFreeVar{n} + 1) \HOLFreeVar{x} \HOLTokenNotEqual{} \HOLFreeVar{n}
\end{alltt}

To prove this, consider first the following lemma equating two evaluations of \texttt{naive\_colouring\_aux} starting from separate values:

\begin{alltt}\small
	\HOLTokenTurnstile{} (\HOLConst{naive_colouring_aux} \HOLFreeVar{cs} (\HOLFreeVar{n} + 1) \HOLFreeVar{x} =
    \HOLConst{naive_colouring_aux} \HOLFreeVar{cs} \HOLFreeVar{n} \HOLFreeVar{x}) \HOLTokenDisj{}
   (\HOLConst{naive_colouring_aux} \HOLFreeVar{cs} (\HOLFreeVar{n} + 1) \HOLFreeVar{x} =
    \HOLConst{naive_colouring_aux} \HOLFreeVar{cs} \HOLFreeVar{n} \HOLFreeVar{x} + 1)
\end{alltt}

This is a clear consequence of the definition: assigned colours starting from $n+1$ are either equal to or one greater than those found by starting from $n$. They are equal in the case where the register is unused (and so is simply mapped to itself by the base case of the definition), and one greater in all other cases. This was shown by induction on the graph. From this we can prove that all assigned colours are greater than the starting value by induction, using the last lemma to link the result of the inductive hypothesis to the goal:

\begin{alltt}\small
	\HOLTokenTurnstile{} \HOLConst{naive_colouring_aux} \HOLFreeVar{cs} (\HOLFreeVar{n} + 1) \HOLFreeVar{x} \HOLTokenGt{} \HOLFreeVar{n}
\end{alltt}

The original lemma is a simple consequence of this:

\begin{alltt}\small
	\HOLTokenTurnstile{} \HOLConst{naive_colouring_aux} \HOLFreeVar{cs} (\HOLFreeVar{n} + 1) \HOLFreeVar{x} \HOLTokenNotEqual{} \HOLFreeVar{n}
\end{alltt}

The second main lemma specifies that updating the mapped value of a function $f$ for a value not featured in $list$ means mapping the updated $f$ over $list$ is the same as mapping $f$ over $list$:

\begin{alltt}\small
	\HOLTokenTurnstile{} \HOLFreeVar{x} \HOLTokenNotIn{} \HOLConst{set} \HOLFreeVar{list} \HOLTokenImp{} (\HOLConst{MAP} ((\HOLFreeVar{x} =+ \HOLFreeVar{n}) \HOLFreeVar{f}) \HOLFreeVar{list} = \HOLConst{MAP} \HOLFreeVar{f} \HOLFreeVar{list})
\end{alltt}

This was proved by inducting on the list and showing that the function's update couldn't affect any element. From this a useful lemma for updating \texttt{naive\_colouring\_aux} was deduced, stating that colouring an unused vertex will not affect the result of applying the colouring:

\begin{alltt}\small
	\HOLTokenTurnstile{} \HOLFreeVar{q} \HOLTokenNotIn{} \HOLConst{set} \HOLFreeVar{r} \HOLTokenImp{}
   (\HOLConst{MAP} ((\HOLFreeVar{q} =+ \HOLFreeVar{n}) (\HOLConst{naive_colouring_aux} \HOLFreeVar{cs} (\HOLFreeVar{n} + 1))) \HOLFreeVar{r} =
    \HOLConst{MAP} (\HOLConst{naive_colouring_aux} \HOLFreeVar{cs} (\HOLFreeVar{n} + 1)) \HOLFreeVar{r})
\end{alltt}

The third lemma to be proved is as follows:

\begin{alltt}\small
	\HOLTokenTurnstile{} (\HOLTokenForall{}\HOLBoundVar{x}. \HOLFreeVar{f} \HOLBoundVar{x} \HOLTokenNotEqual{} \HOLFreeVar{n}) \HOLTokenImp{} \HOLFreeVar{n} \HOLTokenNotIn{} \HOLConst{set} (\HOLConst{MAP} \HOLFreeVar{f} \HOLFreeVar{list})
\end{alltt}

This just states that if a function is never equal to $n$, $n$ will not feature in the result of mapping $f$ over a list. The proof is a simple induction using the definitions of \texttt{MEM} and \texttt{MAP}.

The final lemma claims that a satisfactory naive colouring starting from $n+1$ will still be satisfactory after mapping a register to $n$, as $n$ will not have been used by any other register:

\begin{alltt}\small
	\HOLTokenTurnstile{} \HOLConst{graph_edge_lists_well_formed} \HOLFreeVar{cs} \HOLTokenConj{}
   \HOLConst{colouring_satisfactory} (\HOLConst{naive_colouring_aux} \HOLFreeVar{cs} (\HOLFreeVar{n} + 1)) \HOLFreeVar{cs} \HOLTokenImp{}
   \HOLConst{colouring_satisfactory}
     ((\HOLFreeVar{q} =+ \HOLFreeVar{n}) (\HOLConst{naive_colouring_aux} \HOLFreeVar{cs} (\HOLFreeVar{n} + 1))) \HOLFreeVar{cs}
\end{alltt}

The proof is fairly simple: $n$ is never used, as proved earlier, and intuitively using a colour which has never been used will not cause any clashes. Thus the update is guaranteed to be safe. However, the intuitive statement here was more difficult to prove. A full statement of it is given below:

\begin{alltt}\small
	\HOLTokenTurnstile{} \HOLConst{graph_edge_lists_well_formed} \HOLFreeVar{cs} \HOLTokenConj{}
   \HOLConst{colouring_satisfactory} \HOLFreeVar{c} \HOLFreeVar{cs} \HOLTokenConj{} (\HOLTokenForall{}\HOLBoundVar{x}. \HOLFreeVar{c} \HOLBoundVar{x} \HOLTokenNotEqual{} \HOLFreeVar{n}) \HOLTokenImp{}
   \HOLConst{colouring_satisfactory} ((\HOLFreeVar{q} =+ \HOLFreeVar{n}) \HOLFreeVar{c}) \HOLFreeVar{cs}
\end{alltt}

The proof depends on three lemmas:

\begin{description}
	\item[Outputs cannot exist without being mapped to] \hfill \\
	\begin{alltt}\small
		\HOLTokenTurnstile{} \HOLFreeVar{f} \HOLFreeVar{x} \HOLTokenNotIn{} \HOLConst{set} (\HOLConst{MAP} \HOLFreeVar{f} \HOLFreeVar{list}) \HOLTokenConj{} \HOLFreeVar{f} \HOLFreeVar{x} \HOLTokenNotEqual{} \HOLFreeVar{n} \HOLTokenImp{}
   \HOLFreeVar{f} \HOLFreeVar{x} \HOLTokenNotIn{} \HOLConst{set} (\HOLConst{MAP} ((\HOLFreeVar{q} =+ \HOLFreeVar{n}) \HOLFreeVar{f}) \HOLFreeVar{list})
	\end{alltt}

	This statement is slightly complex, but essentially says that if a value is not in the result of mapping a function over a list and it is not equal to $n$, then it will still not be in the result if we update the function to map something to $n$. The proof depends on the following lemma:

	\begin{alltt}\small
		\HOLTokenTurnstile{} (\HOLFreeVar{q} =+ \HOLFreeVar{n}) \HOLFreeVar{f} \HOLFreeVar{h} \HOLTokenNotEqual{} \HOLFreeVar{f} \HOLFreeVar{h} \HOLTokenImp{} (\HOLFreeVar{h} = \HOLFreeVar{q})
	\end{alltt}

	This can be shown by case splitting on $h = q$ and simplifying. The proof of the original statement now follows using this lemma by induction on the list. We show show that if the head of the list, $h$, contradicts the goal then we must have $(q =+ n) f h = f x$ and thus not equal to $n$, and also $f h \neq f x$. However, this means $(q =+ n) f h \neq f h$ and thus $h = q$ by the lemma. Therefore $f h = n$, contradicting the above.

	\item[Map output only contains values mapped from inputs] \hfill \\
	\begin{alltt}\small
		\HOLTokenTurnstile{} \HOLFreeVar{x} \HOLTokenNotIn{} \HOLConst{set} \HOLFreeVar{list} \HOLTokenConj{} (\HOLTokenForall{}\HOLBoundVar{y}. \HOLBoundVar{y} \HOLTokenNotEqual{} \HOLFreeVar{x} \HOLTokenImp{} \HOLFreeVar{f} \HOLBoundVar{y} \HOLTokenNotEqual{} \HOLFreeVar{n}) \HOLTokenImp{} \HOLFreeVar{n} \HOLTokenNotIn{} \HOLConst{set} (\HOLConst{MAP} \HOLFreeVar{f} \HOLFreeVar{list})
	\end{alltt}

	Simply put, if only one value $x$ maps to a particular output value $n$, mapping the function over a list which does not contain $x$ will give a list not containing $n$. The proof is by induction, and is trivial using the induction hypothesis and expanding the definitions of \texttt{MEM} and \texttt{MAP}.


	\item[Applying an update only changes the updated value] \hfill \\
	\begin{alltt}\small
		\HOLTokenTurnstile{} (\HOLTokenForall{}\HOLBoundVar{x}. \HOLFreeVar{f} \HOLBoundVar{x} \HOLTokenNotEqual{} \HOLFreeVar{n}) \HOLTokenImp{} \HOLTokenForall{}\HOLBoundVar{x}. \HOLBoundVar{x} \HOLTokenNotEqual{} \HOLFreeVar{w} \HOLTokenImp{} (\HOLFreeVar{w} =+ \HOLFreeVar{n}) \HOLFreeVar{f} \HOLBoundVar{x} \HOLTokenNotEqual{} \HOLFreeVar{n}
	\end{alltt}
	
	That is, if a function is never equal to $n$, updating it so one value is mapped to $n$ does not affect any other inputs. The proof is trivial by case splitting on $x = w$ in the conclusion.
\end{description}

These three sub-lemmas are sufficient to prove the lemma. We induct on the graph, and do a case split on whether the current vertex is equal to the one being updated. If it is, we use the fact that the new colour will not have been used before and the current vertex is not in its conflicts list to show that it is satisfactory on the current vertex, and the inductive hypothesis to prove the remainder of the goal. If it is not, we use the fact that the current vertex does not belong to its list of conflicts to show that its colour cannot conflict with anything in the set, and we again use the inductive hypothesis to prove the rest.

It is now possible to prove the original statement of \texttt{naive\_colouring\_aux} being satisfactory:

\begin{alltt}\small
	\HOLTokenTurnstile{} \HOLConst{graph_edge_lists_well_formed} \HOLFreeVar{cs} \HOLTokenImp{}
   \HOLTokenForall{}\HOLBoundVar{n}. \HOLConst{colouring_satisfactory} (\HOLConst{naive_colouring_aux} \HOLFreeVar{cs} \HOLBoundVar{n}) \HOLFreeVar{cs}
\end{alltt}

As mentioned when defining \texttt{colouring\_satisfactory} in Section (TODO defining colouring\_satisfactory), proving a colouring satisfactory consists of two goals: showing that an update doesn't violate the constraints on its vertex, and showing that if the previous colouring was valid for the rest of the graph then the updated one will be too. This latter goal follows from the lemma proved earlier:

\begin{alltt}\small
	\HOLTokenTurnstile{} \HOLConst{graph_edge_lists_well_formed} \HOLFreeVar{cs} \HOLTokenConj{}
   \HOLConst{colouring_satisfactory} \HOLFreeVar{c} \HOLFreeVar{cs} \HOLTokenConj{} (\HOLTokenForall{}\HOLBoundVar{x}. \HOLFreeVar{c} \HOLBoundVar{x} \HOLTokenNotEqual{} \HOLFreeVar{n}) \HOLTokenImp{}
   \HOLConst{colouring_satisfactory} ((\HOLFreeVar{q} =+ \HOLFreeVar{n}) \HOLFreeVar{c}) \HOLFreeVar{cs}
\end{alltt}

The assumption that the original colouring is never equal to $n$ follows from the lemma proved earlier:

\begin{alltt}\small
	\HOLTokenTurnstile{} \HOLConst{naive_colouring_aux} \HOLFreeVar{cs} (\HOLFreeVar{n} + 1) \HOLFreeVar{x} \HOLTokenNotEqual{} \HOLFreeVar{n}
\end{alltt}

The first goal was then proved using the assumption that graph edge lists are well-formed and a combination of the other lemmas proved here.

Thus the verification of \texttt{naive\_colouring\_aux} is complete, and by the implication shown at the start of this section we have that the overall algorithm is correct:

\begin{alltt}\small
	\HOLTokenTurnstile{} \HOLConst{graph_edge_lists_well_formed} \HOLFreeVar{cs} \HOLTokenImp{}
   \HOLConst{colouring_satisfactory} (\HOLConst{naive_colouring} \HOLFreeVar{cs}) \HOLFreeVar{cs}
\end{alltt}

\section{Lowest-first colouring}

The lowest-first algorithm is an improvement over naive colouring which allows colours to be re-used where possible. Essentially, the function examines the vertices of the graph in an arbitrary order and assigns to each the lowest possible colour which does not conflict with the colours already assigned to any of its neighbours. This section explores the definition of this function and the proof of correctness, again using \texttt{colouring\_satisfactory} as the verification condition.

\subsection{The algorithm}

The main algorithm is fairly simple: to colour a particular vertex, colour all those after it in the list and then find the lowest colour which doesn't conflict with the colours that have been assigned to its neighbours. The definition is given here:

\begin{alltt}\small
	\HOLConst{lowest_first_colouring} [] = (\HOLTokenLambda{}\HOLBoundVar{x}. 0)
\HOLConst{lowest_first_colouring} ((\HOLFreeVar{r},\HOLFreeVar{rs})::\HOLFreeVar{cs}) =
(\HOLKeyword{let} \HOLBoundVar{col} = \HOLConst{lowest_first_colouring} \HOLFreeVar{cs} \HOLKeyword{in}
 \HOLKeyword{let} \HOLBoundVar{lowest\HOLTokenUnderscore{}available} = \HOLConst{lowest_available_colour} \HOLBoundVar{col} \HOLFreeVar{rs}
 \HOLKeyword{in}
   (\HOLFreeVar{r} =+ \HOLBoundVar{lowest\HOLTokenUnderscore{}available}) \HOLBoundVar{col})
\end{alltt}

The function \texttt{lowest\_available\_colour} is fairly self-explanatory:

\begin{alltt}\small
	\HOLConst{lowest_available_colour} \HOLFreeVar{col} \HOLFreeVar{cs} =
\HOLConst{smallest_nonmember} 0 (\HOLConst{MAP} \HOLFreeVar{col} \HOLFreeVar{cs})
\end{alltt}

This uses the auxiliary function \texttt{smallest\_nonmember}, which begins from the starting value $x$ and increments until it is not a member of the supplied list, returning the first non-member. The definition is supplied below:

\begin{alltt}\small
	\HOLConst{smallest_nonmember} \HOLFreeVar{x} \HOLFreeVar{list} =
\HOLKeyword{if} \HOLConst{MEM} \HOLFreeVar{x} \HOLFreeVar{list} \HOLKeyword{then} \HOLConst{smallest_nonmember} (\HOLFreeVar{x} + 1) \HOLFreeVar{list} \HOLKeyword{else} \HOLFreeVar{x}
\end{alltt}

\subsubsection{Termination of \texttt{smallest\_nonmember}}

The \texttt{smallest\_nonmember} function's termination is non-trivial, and so had to proved separately. The approach used was to show that each recursive iteration decreased the distance between the smallest nonmember value and the maximum value of the list, found using the following function:

\begin{alltt}\small
	\HOLConst{list_max} [] = 0
\HOLConst{list_max} (\HOLFreeVar{x}::\HOLFreeVar{xs}) =
(\HOLKeyword{let} \HOLBoundVar{tail} = \HOLConst{list_max} \HOLFreeVar{xs} \HOLKeyword{in} \HOLKeyword{if} \HOLFreeVar{x} \HOLTokenGt{} \HOLBoundVar{tail} \HOLKeyword{then} \HOLFreeVar{x} \HOLKeyword{else} \HOLBoundVar{tail})
\end{alltt}

To make the proof easier, the following lemma showing that the maximum of a list is indeed its maximum was proved by induction:

\begin{alltt}\small
	\HOLTokenTurnstile{} \HOLConst{MEM} \HOLFreeVar{x} \HOLFreeVar{list} \HOLTokenImp{} \HOLFreeVar{x} \HOLTokenLeq{} \HOLConst{list_max} \HOLFreeVar{list}
\end{alltt}

It is fairly simple to show that as the value to be returned is incremented on each recursion, it will continuously move towards the maximum of the list, and so termination of \texttt{smallest\_nonmember} can now be proved using a relation which subtracts the value of $x$ from the maximum of the list.

\subsection{Correctness proof}

AAs for the naive colouring, the proof of correctness depends on showing two things: the colour selected for a new vertex will not conflict with any of its neighbours, and the updated colouring will be valid for all vertices whose colours have already been assigned. The first of these amounts to proving that the value returned by \texttt{smallest\_nonmember} is indeed not a member:

\begin{alltt}\small
	\HOLTokenTurnstile{} \HOLConst{smallest_nonmember} \HOLFreeVar{n} \HOLFreeVar{list} \HOLTokenNotIn{} \HOLConst{set} \HOLFreeVar{list}
\end{alltt}

The proof is trivial, using the following custom induction rule generated for \texttt{smallest\_nonmember} and case-splitting on whether $x$ is a member of the list:

\begin{alltt}\small
	\HOLTokenTurnstile{} (\HOLTokenForall{}\HOLBoundVar{x} \HOLBoundVar{list}. (\HOLConst{MEM} \HOLBoundVar{x} \HOLBoundVar{list} \HOLTokenImp{} \HOLFreeVar{P} (\HOLBoundVar{x} + 1) \HOLBoundVar{list}) \HOLTokenImp{} \HOLFreeVar{P} \HOLBoundVar{x} \HOLBoundVar{list}) \HOLTokenImp{}
   \HOLTokenForall{}\HOLBoundVar{v} \HOLBoundVar{v\sb{\mathrm{1}}}. \HOLFreeVar{P} \HOLBoundVar{v} \HOLBoundVar{v\sb{\mathrm{1}}}
\end{alltt}

It is now possible to show that the value selected by \texttt{lowest\_available\_colour} is valid given the set of constraints:

\begin{alltt}\small
	\HOLTokenTurnstile{} \HOLConst{lowest_available_colour} \HOLFreeVar{col} \HOLFreeVar{cs} \HOLTokenNotIn{} \HOLConst{set} (\HOLConst{MAP} \HOLFreeVar{col} \HOLFreeVar{cs})
\end{alltt}

The proof follows by expanding the definition of \texttt{lowest\_available\_colour} and using the previous lemma.

The following lemma was also useful in the proof. It claims that if a colouring update is valid for the current vertex's set of conflicts, and the colouring was valid for all other vertices before updating, then the updated colouring will be valid for all other vertices:

\begin{alltt}\small
	\HOLTokenTurnstile{} \HOLFreeVar{n} \HOLTokenNotIn{} \HOLConst{set} (\HOLConst{MAP} ((\HOLFreeVar{r} =+ \HOLFreeVar{n}) \HOLFreeVar{col}) \HOLFreeVar{rs}) \HOLTokenConj{}
   \HOLConst{colouring_satisfactory} \HOLFreeVar{col} \HOLFreeVar{cs} \HOLTokenImp{}
   \HOLConst{colouring_satisfactory} ((\HOLFreeVar{r} =+ \HOLFreeVar{n}) \HOLFreeVar{col}) ((\HOLFreeVar{r},\HOLFreeVar{rs})::\HOLFreeVar{cs})
\end{alltt}

This intuitively follows from the fact that the list of conflicts for a vertex $v$ should include all vertices which also have $v$ in their conflict lists -- that is, the conflict list for $v$ captures all conflicts it is part of. This covers the second part of the proof, and makes it possible to complete the proof of the algorithm's correctness:

\begin{alltt}\small
	\HOLTokenTurnstile{} \HOLConst{graph_edge_lists_well_formed} \HOLFreeVar{cs} \HOLTokenImp{}
   \HOLConst{colouring_satisfactory} (\HOLConst{lowest_first_colouring} \HOLFreeVar{cs}) \HOLFreeVar{cs}
\end{alltt}

The lowest-first colouring algorithm is therefore correct with respect to \texttt{colouring\_satisfactory}. This is a reasonably good algorithm, but it is still fairly simplistic. In the next section, various heuristics which can further improve performance will be explored.

\section{Heuristics}

The two algorithms developed so far work reasonably well in practice, but can be greatly improved with extra heuristics. These heuristics are used to control the order in which vertices are considered by the colouring algorithm, so that particular vertices are prioritised and others are left as late as possible. In this section the approach to modelling heuristics is explored, and a few real heuristics are examined and verified.

\subsection{Modelling heuristics}

A convenient approach to heuristics, inspired by (TODO ref), is to break the overall algorithm into two steps. The first of these re-orders the list of vertices in the clash graph according to some heuristic, and the second is a colouring stage such as the lowest-first algorithm verified in the last section. This means that a heuristic is correct if and only if the set of vertices after applying the heuristic is the same as the set before. Clash graphs have no duplicate vertices (this follows from the proof that the list of all registers is duplicate-free, given in Section (TODO)), and so converting the set list into a set to test this property will not lose information. This notion is captured succinctly by the following definition:

\begin{alltt}\small
	\HOLConst{heuristic_application_ok} \HOLFreeVar{f} \HOLTokenEquiv{} \HOLTokenForall{}\HOLBoundVar{list}. \HOLConst{set} (\HOLFreeVar{f} \HOLBoundVar{list}) = \HOLConst{set} \HOLBoundVar{list}
\end{alltt}

\subsection{Simple sort-based heuristics}

The simplest kind of heuristic is one which simply sorts elements based on some property -- for instance, a highest-degree-first ordering. This was modelled as a sort with a function passed in which computes the value of this property. The relevant definitions are as follows:

\begin{alltt}\small
	\HOLConst{heuristic_insert} \HOLFreeVar{f} \HOLFreeVar{x} [] = [\HOLFreeVar{x}]
\HOLConst{heuristic_insert} \HOLFreeVar{f} \HOLFreeVar{x} (\HOLFreeVar{y}::\HOLFreeVar{ys}) =
\HOLKeyword{if} \HOLFreeVar{f} \HOLFreeVar{x} \HOLTokenGt{} \HOLFreeVar{f} \HOLFreeVar{y} \HOLKeyword{then} \HOLFreeVar{x}::\HOLFreeVar{y}::\HOLFreeVar{ys} \HOLKeyword{else} \HOLFreeVar{y}::\HOLConst{heuristic_insert} \HOLFreeVar{f} \HOLFreeVar{x} \HOLFreeVar{ys}
	\HOLConst{heuristic_sort} \HOLFreeVar{f} [] = []
\HOLConst{heuristic_sort} \HOLFreeVar{f} (\HOLFreeVar{x}::\HOLFreeVar{xs}) =
\HOLConst{heuristic_insert} \HOLFreeVar{f} \HOLFreeVar{x} (\HOLConst{heuristic_sort} \HOLFreeVar{f} \HOLFreeVar{xs})
\end{alltt}

These implement a simple insertion sort, where \texttt{heuristic\_insert} performs the insertion and \texttt{heuristic\_sort} inserts each element into the list one by one.

To make it easier to verify heuristics based on this approach, a more specific definition of correctness was added:

\begin{alltt}\small
	\HOLConst{sort_heuristic_ok} \HOLFreeVar{f} \HOLTokenEquiv{}
\HOLTokenForall{}\HOLBoundVar{list}. \HOLConst{set} (\HOLConst{heuristic_sort} \HOLFreeVar{f} \HOLBoundVar{list}) = \HOLConst{set} \HOLBoundVar{list}
\end{alltt}

This verifies that sorting with a particular function maintains the members of the original set. This clearly implies the usual definition of heuristic correctness:

\begin{alltt}\small
	\HOLTokenTurnstile{} \HOLConst{sort_heuristic_ok} \HOLFreeVar{f} \HOLTokenImp{}
   \HOLConst{heuristic_application_ok} (\HOLConst{heuristic_sort} \HOLFreeVar{f})
\end{alltt}

The proof follows trivially once the definitions of \texttt{sort\_heuristic\_ok} and \texttt{heuristic\_application\_ok} are expanded.

Intuitively, if this holds for any function $f$ then it should hold for \textit{all} such functions. To prove this, we first verify that \texttt{heuristic\_insert} adds elements to the set correctly:

\begin{alltt}\small
	\HOLTokenTurnstile{} \HOLConst{set} (\HOLConst{heuristic_insert} \HOLFreeVar{f} \HOLFreeVar{h} \HOLFreeVar{list}) = \HOLTokenLeftbrace{}\HOLFreeVar{h}\HOLTokenRightbrace{} \HOLTokenUnion{} \HOLConst{set} \HOLFreeVar{list}
\end{alltt}

This can be proved by induction on the size of the list set to be inserted into, case splitting on whether the element to be inserted will be inserted at the current position or later. It can be used to verify that all sort-based heuristics satisfy \texttt{sort\_heuristic\_ok}, and therefore \texttt{heuristic\_application\_ok} by the implication proved earlier:

\begin{alltt}\small
	\HOLTokenTurnstile{} \HOLConst{sort_heuristic_ok} \HOLFreeVar{f}
\end{alltt}

The proof follows from the correctness of \texttt{heuristic\_insert} by inducting on the size of the list and rearranging set operations using the standard HOL set theory definitions.

As a simple example, consider the following heuristic function which calculates the degree of a vertex:

\begin{alltt}\small
	\HOLConst{vertex_degree} (\HOLFreeVar{v\sb{\mathrm{0}}},[]) = 0
\HOLConst{vertex_degree} (\HOLFreeVar{r},\HOLFreeVar{x}::\HOLFreeVar{xs}) = \HOLConst{vertex_degree} (\HOLFreeVar{r},\HOLFreeVar{xs}) + 1
\end{alltt}

Using this as the sort function with \texttt{heuristic\_sort} will sort vertices according to degree, a common heuristic used when colouring graphs. Its correctness now follows trivially from the theorems proved above.

Many other sort-based heuristics are possible. For instance, if only a finite number of registers are available (a situation which is explored in Section (TODO extensions - finite registers)), we might want to prioritise registers which are frequently accessed for register storage and spill those which are used less often. In this case, the clash graph generator could simply tag vertices with the number of times they are accessed in the code, and this could be used in a heuristic sort to order vertices according to how often they are accessed.

\subsection{More complex heuristics}

TODO


\chapter{Bringing everything together}

In this chapter, the properties verified of the algorithms developed in previous chapters are used to produce an overall proof of correctness for the whole algorithm. The result of this is a high-level proof which shows that applying these register allocation algorithms to a piece of code will yield code with exactly the same evaluation behaviour.

This consists of three main proofs. The first shows that graphs generated by the clash graph generator possess the properties assumed by the colouring algorithms, the second shows that the definition of correctness used for colouring algorithms implies the correctness required for correct evaluation, and the third shows that a colouring satisfying the correctness used for evaluation leads to code whose evaluation behaviour is unchanged.

\section{Linking code correctness to graph correctness}

This section demonstrates that graphs generated by the clash graph generator have the necessary properties assumed by the colouring algorithms, and that a colouring deemed correct with respect to \texttt{colouring\_satisfactory} will also satisfy \texttt{colouring\_ok}. As all colouring algorithms were verified using \texttt{colouring\_satisfactory}, this indicates that they also satisfy \texttt{colouring\_ok} and therefore that the full algorithm generates colourings satisfying \texttt{colouring\_ok} for any code block.

\subsection{Correctness of generated graphs}

For the proofs about graph colouring correctness to be useful, it was necessary to show that the graphs being generated and passed in have the right well-formedness properties -- recall the definitions:

\begin{alltt}\small
	\HOLConst{edge_list_well_formed} (\HOLFreeVar{v},\HOLFreeVar{edges}) \HOLTokenEquiv{}
\HOLFreeVar{v} \HOLTokenNotIn{} \HOLConst{set} \HOLFreeVar{edges} \HOLTokenConj{} \HOLConst{duplicate_free} \HOLFreeVar{edges}
	\HOLConst{graph_edge_lists_well_formed} \HOLFreeVar{es} \HOLTokenEquiv{}
\HOLConst{EVERY} (\HOLTokenLambda{}\HOLBoundVar{x}. \HOLConst{edge_list_well_formed} \HOLBoundVar{x}) \HOLFreeVar{es}
\end{alltt}

Thus the statement to be proved is as follows:

\begin{alltt}\small
	\HOLTokenTurnstile{} \HOLConst{duplicate_free} \HOLFreeVar{live} \HOLTokenImp{}
   \HOLConst{graph_edge_lists_well_formed} (\HOLConst{get_conflicts} \HOLFreeVar{code} \HOLFreeVar{live})
\end{alltt}

We first prove a simpler version of this, which states that the edge list for one vertex is well-formed:

\begin{alltt}\small
	\HOLTokenTurnstile{} \HOLConst{duplicate_free} \HOLFreeVar{live} \HOLTokenImp{}
   \HOLConst{edge_list_well_formed} (\HOLFreeVar{r},\HOLConst{conflicts_for_register} \HOLFreeVar{r} \HOLFreeVar{code} \HOLFreeVar{live})
\end{alltt}

From the definition, we can see that this requires that a register does not appear in its own list of conflicts, and that the list of conflicts is duplicate-free. These two things were proven as lemmas in Section (TODO clash graph generation from conflicting sets):

\begin{alltt}\small
	\HOLTokenTurnstile{} \HOLFreeVar{r} \HOLTokenNotIn{} \HOLConst{set} (\HOLConst{conflicts_for_register} \HOLFreeVar{r} \HOLFreeVar{code} \HOLFreeVar{live})
	\HOLTokenTurnstile{} \HOLConst{duplicate_free} \HOLFreeVar{live} \HOLTokenImp{}
   \HOLConst{duplicate_free} (\HOLConst{conflicts_for_register} \HOLFreeVar{r} \HOLFreeVar{code} \HOLFreeVar{live})
\end{alltt}

These two lemmas are sufficient to prove the simplified goal above. Now that generated graph edge lists have been proven to be individually well-formed, it is relatively simple to prove the original goal:

\begin{alltt}\small
	\HOLTokenTurnstile{} \HOLConst{duplicate_free} \HOLFreeVar{live} \HOLTokenImp{}
   \HOLConst{graph_edge_lists_well_formed} (\HOLConst{get_conflicts} \HOLFreeVar{code} \HOLFreeVar{live})
\end{alltt}

The proof uses one more small lemma, which relates universal quantification of a property to HOL's EVERY function:

\begin{alltt}\small
	\HOLTokenTurnstile{} (\HOLTokenForall{}\HOLBoundVar{x}. \HOLFreeVar{P} (\HOLFreeVar{f} \HOLBoundVar{x})) \HOLTokenImp{} \HOLConst{EVERY} \HOLFreeVar{P} (\HOLConst{MAP} \HOLFreeVar{f} \HOLFreeVar{list})
\end{alltt}

The proof is a trivial induction. With this, the original goal can be proved by simplifying and using the above lemma to put the statement in the form used by the simplified goal.

\subsection{Connecting \texttt{colouring_satisfactory} to \texttt{colouring_ok}}

The next task in the proof was to show that \texttt{colouring\_satisfactory}, the definition of correctness used in verifying graph colouring algorithms, implies \texttt{colouring\_ok}, the definition used when working directly with code. For this it was much more convenient to use the definition \texttt{colouring\_ok\_alt}, which was proved equivalent to \texttt{colouring\_ok} in Section (TODO colouring_ok_alt_def same as colouring_ok_def). The statement to be proved is given below:

\begin{alltt}\small
	\HOLTokenTurnstile{} \HOLConst{duplicate_free} \HOLFreeVar{live} \HOLTokenImp{}
   \HOLConst{colouring_satisfactory} \HOLFreeVar{c} (\HOLConst{get_conflicts} \HOLFreeVar{code} \HOLFreeVar{live}) \HOLTokenImp{}
   \HOLConst{colouring_ok_alt} \HOLFreeVar{c} \HOLFreeVar{code} \HOLFreeVar{live}
\end{alltt}

This ended up being fairly complicated to prove. Initially, a slightly simpler goal was proved which would then be linked to the actual definitions to complete the original proof. The simpler goal is as follows:

\begin{alltt}\small
	\HOLTokenTurnstile{} \HOLConst{duplicate_free} \HOLFreeVar{live} \HOLTokenImp{}
   (\HOLTokenForall{}\HOLBoundVar{r}.
      \HOLFreeVar{col} \HOLBoundVar{r} \HOLTokenNotIn{}
      \HOLConst{set} (\HOLConst{MAP} \HOLFreeVar{col} (\HOLConst{conflicts_for_register} \HOLBoundVar{r} \HOLFreeVar{code} \HOLFreeVar{live}))) \HOLTokenImp{}
   \HOLConst{EVERY} (\HOLTokenLambda{}\HOLBoundVar{s}. \HOLConst{duplicate_free} (\HOLConst{MAP} \HOLFreeVar{col} \HOLBoundVar{s}))
     (\HOLConst{conflicting_sets} \HOLFreeVar{code} \HOLFreeVar{live})
\end{alltt}

This takes the usual assumption that the set of variables specified as live at the end of the code block is duplicate-free. The antecedent here states that for any register $r$ and colouring $col$, applying the colouring to $r$ doesn't conflict with the set of conflicts for $r$ - the coloured $r$ is not a member of the set of conflicts with the colouring mapped over it. As will be demonstrated formally later, this is a consequence of the colouring being satisfactory for the conflicts generated by \texttt{get\_conflicts} - in fact, it is essentially a rephrasing of it. The consequent, meanwhile, states that mapping the colouring over any conflicting set yields a duplicate-free result. This is simply the definition of \texttt{colouring\_respects\_conflicting\_sets} restated using \texttt{EVERY}, and is therefore a restatement of \texttt{colouring\_ok\_alt}. As will be demonstrated later, this goal is sufficient to prove the original goal.

The proof of this statement was fairly complex, making use of several of the lemmas proved earlier. The approach used was to use the following lemma from Section (TODO) to show duplicate-freeness of each conflicting set after colouring:

\begin{alltt}\small
	\HOLTokenTurnstile{} (\HOLTokenForall{}\HOLBoundVar{x} \HOLBoundVar{y}.
      \HOLBoundVar{x} \HOLTokenLt{} \HOLConst{LENGTH} \HOLFreeVar{list} \HOLTokenConj{} \HOLBoundVar{y} \HOLTokenLt{} \HOLConst{LENGTH} \HOLFreeVar{list} \HOLTokenConj{} \HOLBoundVar{x} \HOLTokenNotEqual{} \HOLBoundVar{y} \HOLTokenImp{}
      \HOLConst{EL} \HOLBoundVar{x} \HOLFreeVar{list} \HOLTokenNotEqual{} \HOLConst{EL} \HOLBoundVar{y} \HOLFreeVar{list}) \HOLTokenImp{}
   \HOLConst{duplicate_free} \HOLFreeVar{list}
\end{alltt}

To prove the antecedent of this it was necessary to use several of the lemmas about duplicate-freeness of conflicting sets along with the fact that a list being duplicate-free means no elements are equal:

\begin{alltt}\small
	\HOLTokenTurnstile{} \HOLConst{duplicate_free} \HOLFreeVar{list} \HOLTokenConj{} \HOLFreeVar{x} \HOLTokenLt{} \HOLConst{LENGTH} \HOLFreeVar{list} \HOLTokenConj{} \HOLFreeVar{y} \HOLTokenLt{} \HOLConst{LENGTH} \HOLFreeVar{list} \HOLTokenConj{}
   \HOLFreeVar{x} \HOLTokenNotEqual{} \HOLFreeVar{y} \HOLTokenImp{}
   \HOLConst{EL} \HOLFreeVar{x} \HOLFreeVar{list} \HOLTokenNotEqual{} \HOLConst{EL} \HOLFreeVar{y} \HOLFreeVar{list}
\end{alltt}

The fact that a register's assigned colour is not assigned to any register in its conflicting set (from the antecedent of the overall goal) was then used to show that any two different registers cannot be the same after colouring, as required to show that the conflicting set after colouring is duplicate-free and complete the proof.

This is almost enough to prove the original goal. The next key lemma proves that the antecedent of the last goal is indeed a consequence of a colouring being satisfactory:

\begin{alltt}\small
	\HOLTokenTurnstile{} \HOLConst{colouring_satisfactory} \HOLFreeVar{col} (\HOLConst{get_conflicts} \HOLFreeVar{code} \HOLFreeVar{live}) \HOLTokenImp{}
   \HOLTokenForall{}\HOLBoundVar{r}.
     \HOLFreeVar{col} \HOLBoundVar{r} \HOLTokenNotIn{} \HOLConst{set} (\HOLConst{MAP} \HOLFreeVar{col} (\HOLConst{conflicts_for_register} \HOLBoundVar{r} \HOLFreeVar{code} \HOLFreeVar{live}))
\end{alltt}

To verify this it was first necessary to show that a satisfactory colouring has the desired behaviour on a single vertex:

[colouring_satisfactory_on_one_vertex]
\begin{alltt}\small
	\HOLTokenTurnstile{} \HOLConst{colouring_satisfactory} \HOLFreeVar{c} \HOLFreeVar{cs} \HOLTokenConj{} \HOLConst{MEM} (\HOLFreeVar{r},\HOLFreeVar{rs}) \HOLFreeVar{cs} \HOLTokenImp{}
   \HOLFreeVar{c} \HOLFreeVar{r} \HOLTokenNotIn{} \HOLConst{set} (\HOLConst{MAP} \HOLFreeVar{c} \HOLFreeVar{rs})
\end{alltt}

The proof is a relatively simple induction where the goal follows from the definition of \texttt{colouring\_satisfactory}.

Another small lemma was required, stating that if $x$ is a member of a list then for any colouring $c$, $c x$ is a member of the list after mapping $c$ over it:

\begin{alltt}\small
	\HOLTokenTurnstile{} \HOLConst{MEM} \HOLFreeVar{x} \HOLFreeVar{xs} \HOLTokenImp{} \HOLConst{MEM} (\HOLFreeVar{c} \HOLFreeVar{x}) (\HOLConst{MAP} \HOLFreeVar{c} \HOLFreeVar{xs})
\end{alltt}

The proof is trivial using simplification and evaluation, along with the HOL theorem \texttt{MEM_MAP}:

\begin{alltt}\small
	\HOLTokenTurnstile{} \HOLConst{MEM} \HOLFreeVar{x} (\HOLConst{MAP} \HOLFreeVar{f} \HOLFreeVar{l}) \HOLTokenEquiv{} \HOLTokenExists{}\HOLBoundVar{y}. (\HOLFreeVar{x} = \HOLFreeVar{f} \HOLBoundVar{y}) \HOLTokenConj{} \HOLConst{MEM} \HOLBoundVar{y} \HOLFreeVar{l}
\end{alltt}

Now the goal colouring_satisfactory_expansion can be verified. This makes use of the following lemma which was given in Section (TODO correctness of clash graphs), to avoid having to restrict the statement to registers which are used in the program:

\begin{alltt}\small
	\HOLTokenTurnstile{} \HOLFreeVar{r} \HOLTokenNotIn{} \HOLConst{set} (\HOLConst{get_registers} \HOLFreeVar{code} \HOLFreeVar{live}) \HOLTokenImp{}
   (\HOLConst{conflicts_for_register} \HOLFreeVar{r} \HOLFreeVar{code} \HOLFreeVar{live} = [])
\end{alltt}

The proof works in essence by expanding out the definition of colouring_satisfactory showing that it means any $r$ must satisfy the goal of the statement, but this was fairly complex to prove due to the complexity of colouring_satisfactory and used several of the lemmas proved above.

Thus we have verified that the antecedent of the overall goal implies the antecedent of \texttt{respecting\_register\_conflicts\_respects\_conflicing\_sets}. The full proof can now be derived using these lemmas, expanding the definition of \texttt{colouring\_ok\_alt} to make the goal match up to the consequent of \texttt{respecting\_register\_conflicts\_respects\_conflicting\_sets}.

Therefore all colourings which satisfy colouring_satisfactory will also satisfy \texttt{colouring\_ok\_alt} and thus \texttt{colouring\_ok}. This gives a full end-to-end proof showing that the full register allocation algorithm generates a colouring which satisfies \texttt{colouring\_ok}. The remaining step in proving the project's overall goal is to show that a colouring satisfying \texttt{colouring\_ok} will not affect the behaviour of a block of code, and the proof of this is explained in the next section.

\section{Overall proof of correctness}

Now that the algorithm has been proven to generate colourings satisfying \texttt{colouring\_ok}, it remains to show that \texttt{colouring\_ok} is sufficient to prove that the code behaviour will be unchanged. This statement is as follows:

\begin{alltt}\small
	\HOLTokenTurnstile{} \HOLConst{colouring_ok} \HOLFreeVar{c} \HOLFreeVar{code} \HOLFreeVar{live} \HOLTokenConj{} \HOLConst{no_dead_code} \HOLFreeVar{code} \HOLFreeVar{live} \HOLTokenConj{}
   (\HOLConst{MAP} \HOLFreeVar{s} (\HOLConst{get_live} \HOLFreeVar{code} \HOLFreeVar{live}) =
    \HOLConst{MAP} (\HOLFreeVar{t} o \HOLFreeVar{c}) (\HOLConst{get_live} \HOLFreeVar{code} \HOLFreeVar{live})) \HOLTokenImp{}
   (\HOLConst{MAP} (\HOLConst{eval} \HOLFreeVar{f} \HOLFreeVar{s} \HOLFreeVar{code}) \HOLFreeVar{live} =
    \HOLConst{MAP} (\HOLConst{eval} \HOLFreeVar{f} \HOLFreeVar{t} (\HOLConst{apply} \HOLFreeVar{c} \HOLFreeVar{code}) o \HOLFreeVar{c}) \HOLFreeVar{live})
\end{alltt}

\subsection{The statement of correctness}

There are three main components to the antecedent:

\begin{itemize}
	\item The colouring $c$ satisfies \texttt{colouring\_ok}
	\item There is no dead code -- this will be discussed shortly, but essentially states that there aren't any instructions which modify non-live registers
	\item The two stores $s$ and $t$ agree on the live variables. The second store $t$ is the store after register allocation, hence it is being composed with the colouring $c$ in the statement of store equality
\end{itemize}

The goal to be proved claims the equality of the store with and without colouring after the code has been evaluated. On the left side of the equality, the code is evaluated and the resulting store is mapped over the variables specified as live after the code has executed to get the results. On the right, the colouring is applied to the code and it is then evaluated with the store $t$ ($s$ after register allocation). This is then mapped over the coloured live variables by composing it with $c$ to obtain the equivalent results after colouring. The equality of these indicates that the results of evaluating the code are exactly the same before and after register allocation.

\subsection{Dead code elimination}

Before proving the correctness statement, it was necessary to define a function to eliminate dead code and verify its correctness with \texttt{no\_dead\_code}. The definition of \texttt{no\_dead\_code} is given below:

\begin{alltt}\small
	\HOLConst{no_dead_code} [] \HOLFreeVar{v\sb{\mathrm{0}}} \HOLTokenEquiv{} \HOLConst{T}
\HOLConst{no_dead_code} (\HOLConst{Inst} \HOLFreeVar{w} \HOLFreeVar{r\sb{\mathrm{1}}} \HOLFreeVar{r\sb{\mathrm{2}}}::\HOLFreeVar{code}) \HOLFreeVar{live} \HOLTokenEquiv{}
\HOLConst{MEM} \HOLFreeVar{w} (\HOLConst{get_live} \HOLFreeVar{code} \HOLFreeVar{live}) \HOLTokenConj{} \HOLConst{no_dead_code} \HOLFreeVar{code} \HOLFreeVar{live}
\end{alltt}

The dead code removal function was fairly simple to define, and is given below:

\begin{alltt}\small
	\HOLConst{remove_dead_code} [] \HOLFreeVar{live} = []
\HOLConst{remove_dead_code} (\HOLConst{Inst} \HOLFreeVar{w} \HOLFreeVar{r\sb{\mathrm{1}}} \HOLFreeVar{r\sb{\mathrm{2}}}::\HOLFreeVar{code}) \HOLFreeVar{live} =
(\HOLKeyword{let} \HOLBoundVar{newcode} = \HOLConst{remove_dead_code} \HOLFreeVar{code} \HOLFreeVar{live}
 \HOLKeyword{in}
   \HOLKeyword{if} \HOLConst{MEM} \HOLFreeVar{w} (\HOLConst{get_live} \HOLBoundVar{newcode} \HOLFreeVar{live}) \HOLKeyword{then} \HOLConst{Inst} \HOLFreeVar{w} \HOLFreeVar{r\sb{\mathrm{1}}} \HOLFreeVar{r\sb{\mathrm{2}}}::\HOLBoundVar{newcode}
   \HOLKeyword{else} \HOLBoundVar{newcode})
\end{alltt}

The proof of correctness was relatively easy, using induction and case splitting on whether the current instruction is dead. The correctness statement is as follows:

\begin{alltt}\small
	\HOLTokenTurnstile{} \HOLConst{no_dead_code} (\HOLConst{remove_dead_code} \HOLFreeVar{code} \HOLFreeVar{live}) \HOLFreeVar{live}
\end{alltt}

\subsection{Lemmas}

The correctness proof depends on a number of lemmas. Firstly, the proof was by induction and so making use of the inductive hypothesis required a pair of proofs showing that \texttt{colouring\_ok} and \texttt{no\_dead\_code} are preserved when the first instruction is removed. The proofs of these statements were trivial by expanding definitions:

\begin{alltt}\small
	\infer{\HOLinline{\HOLConst{colouring_ok} \HOLFreeVar{c} \HOLFreeVar{code}
  \HOLFreeVar{live}}}{\HOLinline{\HOLConst{colouring_ok} \HOLFreeVar{c} (\HOLConst{Inst} \HOLFreeVar{n} \HOLFreeVar{n\sb{\mathrm{0}}} \HOLFreeVar{n\sb{\mathrm{1}}}::\HOLFreeVar{code}) \HOLFreeVar{live}}}
	\infer{\HOLinline{\HOLConst{no_dead_code} \HOLFreeVar{code} \HOLFreeVar{live}}}{\HOLinline{\HOLConst{no_dead_code} (\HOLConst{Inst} \HOLFreeVar{n} \HOLFreeVar{n\sb{\mathrm{0}}} \HOLFreeVar{n\sb{\mathrm{1}}}::\HOLFreeVar{code}) \HOLFreeVar{live}}}
\end{alltt}

Another small yet useful lemma is the following, stating that \texttt{colouring\_ok} means applying the colouring to the current set of live variables yields a duplicate-free result:

\begin{alltt}\small
	\infer{\HOLinline{\HOLConst{duplicate_free}
  (\HOLConst{MAP} \HOLFreeVar{c} (\HOLConst{get_live} \HOLFreeVar{code} \HOLFreeVar{live}))}}{\HOLinline{\HOLConst{colouring_ok} \HOLFreeVar{c} \HOLFreeVar{code} \HOLFreeVar{live}}}
\end{alltt}

This was proved by case splitting on the code variable then evaluating and simplifying the result to prove the goal.

It was also useful to be able to state that if the result of mapping a colouring over a list is duplicate-free and one considers two unequal elements of the original list, they will still not be equal after applying the function to each of them:

\begin{alltt}\small
	\HOLTokenTurnstile{} \HOLConst{duplicate_free} (\HOLConst{MAP} \HOLFreeVar{c} \HOLFreeVar{live}) \HOLTokenConj{} \HOLFreeVar{x} \HOLTokenNotEqual{} \HOLFreeVar{y} \HOLTokenConj{} \HOLConst{MEM} \HOLFreeVar{x} \HOLFreeVar{live} \HOLTokenConj{}
   \HOLConst{MEM} \HOLFreeVar{y} \HOLFreeVar{live} \HOLTokenImp{}
   \HOLFreeVar{c} \HOLFreeVar{x} \HOLTokenNotEqual{} \HOLFreeVar{c} \HOLFreeVar{y}
\end{alltt}

This was proved in Section (TODO).

These last two lemmas make it possible to prove the following statement, which essentially states that a colouring satisfying \texttt{colouring\_ok} for a particular block of code is injective on the set of live variables for that code:

\begin{alltt}\small
	\HOLTokenTurnstile{} \HOLConst{no_dead_code} \HOLFreeVar{code} \HOLFreeVar{live} \HOLTokenConj{} \HOLConst{colouring_ok} \HOLFreeVar{c} \HOLFreeVar{code} \HOLFreeVar{live} \HOLTokenConj{} \HOLFreeVar{x} \HOLTokenNotEqual{} \HOLFreeVar{y} \HOLTokenConj{}
   \HOLConst{MEM} \HOLFreeVar{x} (\HOLConst{get_live} \HOLFreeVar{code} \HOLFreeVar{live}) \HOLTokenConj{} \HOLConst{MEM} \HOLFreeVar{y} (\HOLConst{get_live} \HOLFreeVar{code} \HOLFreeVar{live}) \HOLTokenImp{}
   \HOLFreeVar{c} \HOLFreeVar{x} \HOLTokenNotEqual{} \HOLFreeVar{c} \HOLFreeVar{y}
\end{alltt}

This follows trivially from the two lemmas above.

\subsection{Proof of correctness}

It is now possible to complete the proof of the original statement of correctness:

\begin{alltt}\small
	\HOLTokenTurnstile{} \HOLConst{colouring_ok} \HOLFreeVar{c} \HOLFreeVar{code} \HOLFreeVar{live} \HOLTokenConj{} \HOLConst{no_dead_code} \HOLFreeVar{code} \HOLFreeVar{live} \HOLTokenConj{}
   (\HOLConst{MAP} \HOLFreeVar{s} (\HOLConst{get_live} \HOLFreeVar{code} \HOLFreeVar{live}) =
    \HOLConst{MAP} (\HOLFreeVar{t} o \HOLFreeVar{c}) (\HOLConst{get_live} \HOLFreeVar{code} \HOLFreeVar{live})) \HOLTokenImp{}
   (\HOLConst{MAP} (\HOLConst{eval} \HOLFreeVar{f} \HOLFreeVar{s} \HOLFreeVar{code}) \HOLFreeVar{live} =
    \HOLConst{MAP} (\HOLConst{eval} \HOLFreeVar{f} \HOLFreeVar{t} (\HOLConst{apply} \HOLFreeVar{c} \HOLFreeVar{code}) o \HOLFreeVar{c}) \HOLFreeVar{live})
\end{alltt}

The proof is by induction on the block of code, and uses all the lemmas above along with various definition expansions and some of the HOL lemmas regarding \texttt{MAP} and \texttt{MEM}. With this proved, it follows that the full register allocation algorithm is correct in that it doesn't affect code behaviour with respect to evaluation.


\chapter{Extension features}

TODO


\chapter{Summary and conclusions}

TODO




\appendix
\singlespacing

\bibliographystyle{unsrt} 
%\bibliography{dissertation} 

\end{document}
